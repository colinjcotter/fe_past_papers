\documentclass[11pt]{article}
\usepackage{graphicx, amssymb}
\usepackage{amsfonts}
\usepackage{epsfig}
\usepackage{amssymb,amsmath,bm}
%\usepackage{subfig}
\usepackage{amsthm}
\usepackage{mathrsfs,mathtools}


%%%%%%%%%%%%%%%%%%%%%%%%%%%%%%%%%%%
%  Fill in these details
\newcommand{\coursenum}{XXX} %{M2S1}
\newcommand{\coursename}{XXX (Solutions)} 
\renewcommand{\pagetotal}{\pageref{pagetot}}     % adjust after viewing draft 
\newcommand{\examyear}{2021}%Please update 
\newcommand{\exammonth}{May} %Please update

%%%%%%%%%%%%%%%%%%%%%%%%%%%%%%%%

\pretolerance=100000
 \setlength{\topmargin}{-3mm}
 \setlength{\textheight}{250mm}


%%%%%%%%%%%%%%%%%%%%%%%%%%%%%%%%%%%%%%%% 
%Added the following
\setlength{\footskip}{0mm}
\setlength{\headsep}{0mm}
\setlength{\headheight}{0mm}



%%%%%%%%%%%%%%%%%%%%%%%%%%%%%%%%

\setlength{\textwidth}{150mm}
\addtolength{\oddsidemargin}{-20mm}
\addtolength{\evensidemargin}{40mm}
\setlength{\marginparwidth}{2cm}
\setlength{\parskip}{1.1ex}
\setlength{\parindent}{0ex}
\renewcommand{\baselinestretch}{1.1}

% seen, unseen and similar seen commands; add in your own as appropriate
\newcommand{\marb}[1]{\marginpar{~~\fbox{\,\mathstrut #1\,}}}  
\newcommand{\marbup}[1]{\vspace*{-4mm} \par \marb{#1}} 
\newcommand{\seen}{\marbup{{\small seen $\Downarrow$}}} 
\newcommand{\unseen}{\marbup{{\small unseen $\Downarrow$}}} 
\newcommand{\simseen}{\marbup{{\small sim. seen $\Downarrow$}}} 
\newcommand{\seensim}{\marbup{{\small seen/sim.seen $\Downarrow$}}} 
\newcommand{\partseen}{\marbup{{\small part seen $\Downarrow$}}} 
\newcommand{\methodseen}{\marbup{{\small meth seen $\Downarrow$}}} 

\pagestyle{empty}
\newcommand{\vect}[1]{\mbox{\boldmath{$#1$}}}

\newcommand{\pp}[2]{\frac{\partial #1}{\partial #2}} 
\newcommand{\dede}[2]{\frac{\delta #1}{\delta #2}}
\newcommand{\dd}[2]{\frac{\diff#1}{\diff#2}}
\newcommand{\dt}[1]{\diff\!#1}
\def\MM#1{\boldsymbol{#1}}
\DeclareMathOperator{\diff}{d}


% setup a counters to count up marks in a question
\newcounter{count_marks}
\setcounter{count_marks}{0}
\newcounter{count_amarks}
\setcounter{count_amarks}{0}
\newcounter{count_bmarks}
\setcounter{count_bmarks}{0}
\newcounter{count_cmarks}
\setcounter{count_cmarks}{0}
\newcounter{count_dmarks}
\setcounter{count_dmarks}{0}
\newcounter{count_mmarks} %Mastery marks
\setcounter{count_mmarks}{0} %Mastery marks
% Define the command for formatting the marks available for the question
\newcommand{\amarks}[1]{\addtocounter{count_marks}{#1} \addtocounter{count_amarks}{#1}\marginpar{~~\fbox{\,\mathstrut #1, A}}}  

\newcommand{\bmarks}[1]{\addtocounter{count_marks}{#1} \addtocounter{count_bmarks}{#1}\marginpar{~~\fbox{\,\mathstrut #1, B}}}  

\newcommand{\cmarks}[1]{\addtocounter{count_marks}{#1} \addtocounter{count_cmarks}{#1}\marginpar{~~\fbox{\,\mathstrut #1, C}}}  

\newcommand{\dmarks}[1]{\addtocounter{count_marks}{#1} \addtocounter{count_dmarks}{#1}\marginpar{~~\fbox{\,\mathstrut #1, D}}}  

\newcommand{\mmarks}[1]{\addtocounter{count_marks}{#1} \addtocounter{count_mmarks}{#1}\marginpar{~~\fbox{\,\mathstrut #1, M}}}  %Mastery marks


% define a total marks command for the end of the question
\newcommand{\totamarks}{\hfill{Total A marks: \arabic{count_amarks} of 
%24 %3 questions
%48 %6 questions
32 %4 questions
 marks}}
\newcommand{\totbmarks}{\hfill{Total B marks: \arabic{count_bmarks} of 
%15 %3 questions
% 30 % 6 questions
20 %4 questions
%Please adjust the number accordingly
marks}}
\newcommand{\totcmarks}{\hfill{Total C marks: \arabic{count_cmarks} of 
%9 %3 questions
% 18 %6 questions
12 %4 questions
%Please adjust the number accordingly
marks}}
\newcommand{\totdmarks}{\hfill{Total D marks: \arabic{count_dmarks} of 
%12 % 3 questions
% 24 % 6 questions
16 % 4 questions
%Please adjust the number accordingly
marks}}

\newcommand{\totmarks}{\hfill{Total marks: \arabic{count_marks} of 
%60 % 3 questions
%120 % 6 questions
80 %4 questions
%Please adjust the number accordingly
marks}}

\newcommand{\totMmarks}{\hfill{Total Mastery marks: \arabic{count_mmarks} of 
20  marks}}

\newenvironment{Question}[1] 
 {\begin{itemize} \item[\large #1.~~]}{\end{itemize}\vfill
} 
%\makeatletter
\newenvironment{Part}[1] 
 {\begin{itemize} \item[(#1)~~]}{\end{itemize}}
%\makeatother
\newcommand{\EndPage}{\vfill \coursenum ~ \coursename ~
 (\examyear) \hfill Page \thepage\/ of \pagetotal \newpage}

%\newcommand{\ICLOGO}{\begin{figure}[!t]
%  \includegraphics[totalheight= 1.1cm] {imperial.eps} 
%  \end{figure} \par }
%%%%
% This requires imperial.eps in the graphics input path eg 
% h:/images/imperial.eps 
%%%% 


\newcommand{\draft}{\begin{flushright} \begin{tabular}{ll}
  Course:   & \coursenum\\ Setter:   & \setter \\
  Checker:  & \checker \\  Editor:   & \editor \\
  External: & \external \\ Date:     & \today
  \end{tabular} \end{flushright} \vfill \par}

\newcommand{\sigs}{\vfill \par
  \fbox{\begin{minipage}{0.98\textwidth}{~ \\[4mm]
  \hspace*{3mm} Setter's signature \hfill
  Checker's signature \hfill
  Editor's signature~~~~ \\[4mm]
  \hspace*{3mm} \dotfill \hfill \dotfill \hfill \dotfill ~~~~
  \\[2mm] ~} \end{minipage}}}

\newcommand{\fpageDraft}  % FOR DRAFT FRONT PAGE
{\sf % \ICLOGO
 \begin{center} IMPERIAL COLLEGE LONDON \par
 \draft
 BSc and MSci EXAMINATIONS (MATHEMATICS) \par 
 \exammonth~ \examyear \vfill \par
 \Large \coursenum \bigskip \par
 \coursename \bigskip \par
 \end{center} \vfill \par \normalsize
 \sigs  \par \EndPage}

\newcommand{\fpage}
{\sf %\ICLOGO
  \begin{center} %UNIVERSITY OF LONDON \par
  BSc and MSci EXAMINATIONS (MATHEMATICS) \par
  \exammonth~ \examyear \bigskip \par
  This paper is also taken for the relevant examination
  for the Associateship. 
  \vfill \par
  \Large \coursenum \bigskip \par
  \coursename  \normalsize \bigskip \par \vfill
  \sigs\end{center}
  \par \vfill \vfill
  \copyright ~\examyear~ Imperial College London
  \hfill \coursenum \hfill Page 1 of \pagetotal
  \newpage}

% Personally entered commands:


% use \begin{Question}{1} ... \end{Question} and
%     \begin{Part}{a} ... \end{Part} 
% use \marb{} to allocate marks and \seen \unseen etc. 
% manually put in \EndPage for correct footers.

% document text starts here
\begin{document}

\setcounter{page}{1}

 \fpage
 

 \sf

%%%%%%%%%%%%%%%%%%%%%%%%%%%%%%%
\begin{Question}{1}
\begin{Part}{a}\simseen
  The Vandermonde matrix and its inverse are
  \begin{equation}
    V=
    \begin{pmatrix}
      1 & 1 & 1 & 1 \\
      0 & 1 & 0 & 1 \\
      0 & 0 & 1 & 1 \\
      0 & 0 & 0 & 1 \\
    \end{pmatrix},\quad
    V^{-1}=
    \begin{pmatrix}
      1 & -1 & -1 & 1 \\
      0 & 1 & 0 & -1 \\
      0 & 0 & 1 & -1 \\
      0 & 0 & 0 & 1 \\
    \end{pmatrix},
  \end{equation}
  which can be computed by hand by e.g. doing back-substitution on the
  columns of the identity matrix. The basis is then
  \begin{align}
    \psi_1(x,y) &= 1-x-y+xy = (1-x)(1-y), \\
    \psi_2(x,y) &= x-xy = x(1-y), \\
    \psi_3(x,y) &= y-xy = y(1-x), \\
    \psi_4(x,y) &= xy.
  \end{align}
  \amarks{8}
\end{Part}

\begin{Part}{b}\unseen
  \begin{Part}{i}
    Suitable nodal variables are $N_i(p)=p(z_i)$ where $z_1=(1,0)$,
    $z_2=(0,1)$, $z_3=(0,0)$, $z_4(1/3,1/3)$.\\ We can check that the
    basis is a nodal one for these nodes by noticing that the spanning
    set for $P$ is the linear functions plus a cubic ``bubble''
    function that vanishes on the triangle vertices (and the edges).
    Thus to make a nodal basis for our nodal variables, the basis
    functions 1 to 3 can just be the usual linear basis functions
    (that are equal to 1 on the corresponding vertex and 0 at all the
    others) plus a scalar multiple of the bubble function so that they
    vanish at $z_4$. The bubble vanishes at all the vertices
    so it just needs to be scaled appropriately to take the value 1 at
    $z_4$ as required.
    \amarks{6}\\
  \end{Part}
\begin{Part}{ii}\simseen
  We assign $N_i$ to vertex $z_i$ for $i=1,2,3$, and the bubble
  function the entire cell. This is a C0 geometric decomposition
  because: (1) the value at each vertex can be obtained from the nodal
  variable assigned to that vertex (since it is just point evaluation
  at the vertex), (2) the value at each edge can be obtained from the
  nodal variables assigned to the closure of the edge, which is just
  vertex values at each end in this case, and the function is linear
  when restricted to an edge.
  \bmarks{6}
\end{Part}
\end{Part}
\end{Question}
	


\EndPage
%%%%%%%%%%%%%%%%%%%%%%%%%%%%%%%%%%%%%%%%%%%%%%%%%%%%%%%%%%%
\begin{Question}{2}
  \begin{Part}{a}\unseen
    The theorem is insufficient because $|u|_{H^{3}(K_1)}$ is
    unbounded, so it doesn't provide any bound on the error.
    \bmarks{6}
  \end{Part}
\begin{Part}{b}\simseen
  The appropriate statement is (under the same conditions as
  5.28 but with $u\in H^2(K_1)$, $k=3$),
  \begin{equation}
    |\mathcal{I}_{K_1}u-u|_{H^1(K_1)} \leq C_1|u|_{H^2(K_1)}.
  \end{equation}
  To prove it,
  \begin{align}
    |\mathcal{I}_{K_1}u-u|_{H^1(K_1)} &\leq \|Q_{3,B}u-u\|^2_{H^1(K_1)}
    + \|\mathcal{I}_{K_1}(u-Q_{3,B}u)\|^2_{H^1(K_1)} \\
    & \leq (1+C^2)|u|^2_{H^2(K_1)},
  \end{align}
  where $Q_{3,B}$ is the degree $k$ averaged Taylor polynomial over
  a ball $B$ inside $K_1$ but as large as possible, and where
  we used Lemmas 3.22 and Corollary 3.16.
\bmarks{7}
\end{Part}
\begin{Part}{c}
  The appropriate statement is (under the same conditions
  as 5.30 but with $u\in H^2(\Omega)$)
  \begin{equation}
    |\mathcal{I}_Ku-u|_{H^1(\Omega)} \leq Ch|u|_{H^2(\Omega)}.
  \end{equation}
  To show this, note that we can obtain the local estimate
  \begin{equation}
    |\mathcal{I}_Ku-u|_{H^1(K)} \leq C_Kd|u|_{H^{k+1}(K)},
  \end{equation}
  by following the steps in the proof but with $k$ replaced by 1.
  Then the same technique of summing over all the cells gives
  the global result.
  \cmarks{7}
 \end{Part}
\end{Question}
\EndPage

\begin{Question}{3}
  \begin{Part}{a} \simseen
    To derive the variational form, we multiply by a test function $v$
    and integrate by parts as usual to get
    \begin{equation}
      \int_{\Omega}\nabla v \cdot \nabla u \diff x
      - \int_{\partial\Omega} v\underbrace{\pp{u}{n}}_{=g}\diff S = 0,
    \end{equation}
    so a suitable variational form is to find $v\in \bar{V}_h$ such that
    \begin{equation}
      \int_{\Omega} \nabla v \cdot \nabla u \diff x
      = \int_{\partial\Omega} vg\diff S, \quad \forall v \in \bar{V}_h,
    \end{equation}
    where $\bar{V}_h$ is the subspace of $V_h$ of functions that
    integrate to zero, and $V_h$ is some choice of $C^0$ finite element
    space.
 \amarks{6}
\end{Part}

\begin{Part}{b} \unseen
  The issue is that the integrals are not tractable in general, so we
  can't evaluate the RHS of the problem. A possible modification is
  to interpolate $g$ to $V_h$ in the boundary resulting in $g_h$, and solve the perturbed
  problem
  \begin{equation}
    \int_{\Omega} \nabla v \cdot \nabla u \diff x
    = \int_{\partial\Omega} vg_h\diff S, \quad \forall v \in \bar{V}_h.
  \end{equation}
 \dmarks{6}
\end{Part}
\begin{Part}{c} \unseen
  The modification to C\'ea's Lemma is
  \begin{align}
    \|u-u_h\|_{H^1(\Omega)} &\leq
    (1+M/\gamma)\inf_{v\in V_h}\|u-v\|_{H^1(\Omega)} +
    \frac{C}{\gamma}\|g-g_h\|_{L^2(\partial\Omega)},
  \end{align}
  so there are now two terms, a best approximation term of $u$ in $V_h$,
  and an approximation error term for $g_h$.

  
  To prove it, following the steps of C\'ea's Lemma, we take a test
  function $v\in V_h$ in both the exact and approximate equation, and
  compute the difference, to yield
  \begin{equation}
    a(u-u_h,v) = \int_{\partial \Omega} v(g-g_h) \diff S, \,
    \forall v \in V_h.
  \end{equation}
  Then we use coercivity to write (for any $v\in V_h$)
  \begin{align}
    \gamma\|u_h-v\|_{H^1(\Omega)} &\leq
    a(u_h-v,u_h-v), \\
    &= a(u_h-u,u_h-v) + a(u-v,u_h-v), \\
    &= \int_{\partial\Omega}(u_h-v)(g_h-g)\diff S
    + a(u-v,u_h-v), \\
    &\leq C\|u_h-v\|_{H^1(\Omega)}\|g_h-g\|_{L^2(\partial\Omega)}
    + M\|u-v\|_{H^1(\Omega)}\|u_h-v\|_{H^1(\Omega)},
  \end{align}
  where $C$ is the constant in the trace inequality and $M$
  is the continuity constant of the bilinear form $a(u,v)$.
  Then, dividing by $\|u_h-v\|_{H^1(\Omega)}$ gives
  \begin{equation}
    \gamma\|u_h-v\|_{H^1(\Omega)}^2 \leq C\|g_h-g\|_{L^2(\partial\Omega)}
    + M\|u-v\|_{H^1(\Omega)}.
  \end{equation}
  Then, combining with the triangle inequality, we get
  \begin{align}
    \|u-u_h\|_{H^1(\Omega)} &\leq \|u-v\|_{H^1(\Omega)} +
    \|u_h-v\|_{H^1(\Omega)}, \\ &\leq
    (1+M/\gamma)\|u-v\|_{H^1(\Omega)} +
    \frac{C}{\gamma}\|g-g_h\|_{L^2(\partial\Omega)},
  \end{align}
  and minimisation over $v$ gives the result.
 \dmarks{8}
\end{Part}

\end{Question}
\EndPage

%%%%%%%%%%%%%%%%%%%%%%%
\begin{Question}{4}
\begin{Part}{a} \unseen
  Multiplying by a test function $v$ that vanishes on the exterior
  boundary and integrating by parts separately in $\Omega_1$ and $\Omega_2$
  gives
  \begin{equation}
    \int_{\Omega} \nabla v\cdot \nabla u \diff x
    - \int_{\Gamma}    v\left(\frac{\partial u}{\partial n}|_{\partial\Omega_1}
    + \frac{\partial u}{\partial n}|_{\partial\Omega_2\cap \Gamma}\right)\diff S
    = 0,
  \end{equation}
  and substitution of the boundary condition gives the variational
  problem: find $u_h\in V_h$ such that
  \begin{equation}
      \int_{\Omega} \nabla v\cdot \nabla u_h \diff x
      = 2\int_{\Gamma}v\diff S, \quad \forall v\in V_h.
  \end{equation}
    \amarks{6}
\end{Part}

\begin{Part}{b} \simseen
  We are in the case of Theorem 4.38, so we just need to check continuity
  of the linear form according to
  \begin{equation}
    F[v] = 2\int_{\Gamma} v \diff S
    \leq \|v\|_{L^2(\Gamma)}2|\Gamma|
    \leq \|v\|_{H^1(\Omega_0)}2|\Gamma|
    \leq \|v\|_{H^1(\Omega)}2|\Gamma|.
  \end{equation}
  where we have used the trace theorem for continuous finite elements
  (Theorem 4.4), and $|\Gamma|=\int_{\Gamma}\diff S$. Hence $F$ is continuous.
  \amarks{8}
\end{Part}

\begin{Part}{c} \unseen
  The bound studied in the course is
  \begin{equation}
    \|u_h-u\|_{H^1(\Omega)} \leq Ch|u|_{H^2(\Omega)},
  \end{equation}
  but the solution has a jump in the first derivative across $\Gamma$,
  so $|u|_{H^2(\Omega)}$ is not finite, so the bound does not
  imply convergence of the numerical solution as $h\to 0$.
\end{Part}
    \cmarks{6}
\end{Question}

\EndPage

%%%%%%%%%%%%%%%%%%%%%%%
\begin{Question}{5}
  \begin{Part}{a} \unseen
    The map is surjective, so there exists
    $v\in V$ such that $q=\nabla\cdot v$ for all $q\in Q$.
    Hence, using the Riesz Representation Theorem, for all
    $F\in Q'$, there exists $q_F$ such that
    \begin{equation}
      F[p] = \int_{\Omega} pq_F\diff x, \quad \forall p\in Q.
    \end{equation}
    So, for all $F\in Q'$, there exists $v\in V$ such that
    \begin{equation}
      b(v,p) = \int_{\Omega}\nabla \cdot v p\diff x
      = F[p], \quad \forall p\in Q.
    \end{equation}
    In other words, for all $F\in Q'$ there exists $v$
    such that $Bv=F$, which means that $B$ is surjective.
    Then, from the notes, this implies the inf-sup condition.
    \mmarks{5}
  \end{Part}

  \begin{Part}{b} \unseen
    Let $q\in \mathrm{Ker}(\delta)$, i.e. $\delta q=0$. Taking
    $w$ such that $\nabla\cdot w=q$, we have
    \begin{equation}
      0 = \int_\Omega \nabla\cdot wq\diff x
      = \int_\Omega q^2 \diff x \implies q = 0.
    \end{equation}
    \mmarks{5}
  \end{Part}

  \begin{Part}{c} \unseen
    Let $q\in \mathrm{Ker}(\delta_h)$. Then for $w\in V$,
    \begin{align}
      \int_\Omega w\cdot \delta q \diff x &=
      b(w,q) = b(\Pi_hw,q), \\
      & = \int_\Omega \Pi_hw\cdot \delta_h q\diff x = 0,
    \end{align}
    so $q\in \mathrm{Ker}(\delta)$ as required.
    \mmarks{5}
  \end{Part}

  \begin{Part}{d} \unseen
    Considering $p\in Q_h$ having a pattern of the type given in Figure
    2, it suffices to consider $b(w,p)$ for $w$ being basis functions
    associated with vertices in the interior of the mesh, which
    span $V_h$ (because of the zero boundary condition).
    The support of $w$ consists of 6 triangles with symmetry about
    a diagonal line from top-left to bottom-right. The divergence
    of $w$ is symmetric about that line, whilst $p$ is antisymmetric,
    so the integral $b(w,p)$ vanishes. Hence, $p\in \ker(\delta_h)$.
    The previous result says that the Fortin Trick assumptions imply
    that the $\ker(\delta_h)$ is empty, $V_h,Q_h$ must fail to satisfy
    the Fortin Trick assumptions.
        \mmarks{5}
  \end{Part}
\end{Question}

\EndPage
{\bf Review of mark distribution:}\\ 
\totamarks\\
 \totbmarks\\
 \totcmarks\\
 \totdmarks\\
  \totmarks
  \\  \totMmarks %Include Mastery marks
 \label{pagetot}
\EndPage


%%%%%%%%%%%%%%%%%%%%%%%%%%%%%%%%%%
\end{document}




