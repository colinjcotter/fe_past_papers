
\documentclass[12pt]{article}

\usepackage{graphicx,amssymb,amsmath,nicefrac,ifthen,calc}
\usepackage[T1]{fontenc}
\usepackage{lmodern}

%%%%%%%%%%%%%%%%%%%%%%%%%%%%%%%%%%%
%  Fill in the following details  %
%%%%%%%%%%%%%%%%%%%%%%%%%%%%%%%%%%%

\newcommand{\coursenum}{M3A47, M4A47, M5A47} %{e.g: M1S, or: M3S14/M4S14/M5S15 for multiple year courses}
\newcommand{\coursename}{Finite elements: analysis and implementation} %{e.g: Probability and Statistics} 
\newcommand{\setter}{Cotter}     % Insert surname of exam setter 
\newcommand{\checker}{Ham} % Insert surname of exam checker 
\newcommand{\editor}{Wu} % Insert surname of exam editor
\newcommand{\external}{external} % Insert surname of external examiner
\newboolean{tables}
\setboolean{tables}{false} %set to true if statistical tables are provided and attach require tables

% Choose an option for time allowed and number of questions %

\newcommand{\examlength}%Uncomment one of the following 3 options, or adjust.
%{2~Hours}
%{2.5~Hours}
{2 Hours for M3 paper; 2.5 Hours for M4/5 paper}

\newcommand{\numQ}%Uncomment one of the following 3 options, or adjust.
%{4~Questions}
%{5~Questions}
{4~Questions (M3 version); 5~Questions (M4/5~version)}

\newcommand{\version}% Uncomment one of the following, or adjust
{Draft version for checking}
%{Version for External Examiner}
%{Final version}


\newcommand{\suitability}% Uncomment one of the options below
{Is the paper suitable for resitting students from previous years:
Yes
  %No (provide details)
}



%%%%%%%%%%%%%%%%%%%%%%%
%The following details about the difficulty level of the questions must be completed; also include this information on your mark-scheme.
%%%%%%%%%%%%%%%%%%%%%%%

\newcommand{\CatA}% Indicate, with marks, the most straightforward 40 percent of the marks available in the non-mastery questions
{1a (5 marks), 2a (5 marks), 2b.i (5 marks), 2b.ii (5 marks), 3b (5 marks), 4a (5 marks) = 35 marks = 43.75\%}
\newcommand{\CatB}% Indicate the next most challenging 25 percent of the marks in the non-mastery questions
{1a 2b (8 marks), 4b+c (10 marks) = 18 marks = 22.5\%}
\newcommand{\CatC}%Indicate the next most challenging 15 percent of the marks in the non-mastery questions
{1b (10 marks), 2a (8 marks) = 18 marks = 22.5\%}
\newcommand{\CatD}% Indicate the most challenging 20 percent of marks on the paper in the non-mastery questions
{1c (5 marks), 4d (5 marks), 2c (4 marks) = 14 marks = 17.5\%}


%%%%%%%%%%%%%%%%%%%%%%%%%%%%%%%%%%%%%
% Insert your macros %
%%%%%%%%%%%%%%%%%%%%%%%%%%%%%%%%%%%%%

\newcommand{\seen}{{\bfseries SEEN\\}}
\newcommand{\similar}{{\bfseries SEEN SIMILAR\\}}
\newcommand{\unseen}{{\bfseries UNSEEN\\}}
\newcommand{\exammarks}[1]{\begin{flushright}[#1 marks]\end{flushright}}%
\newcommand{\pp}[2]{\frac{\partial #1}{\partial #2}} 
\newcommand{\dede}[2]{\frac{\delta #1}{\delta #2}}
\newcommand{\dd}[2]{\frac{\diff#1}{\diff#2}}
\newcommand{\dt}[1]{\diff\!#1}
\def\MM#1{\boldsymbol{#1}}
\DeclareMathOperator{\diff}{d}


%uncomment \solnsfalse to remove solution set
\newif\ifsolns
%\solnstrue
%\solnsfalse
\input{flags}
\ifsolns
% with solutions
\usepackage{color}
\newcommand{\soln}[1]{{\bfseries Solution:} {\itshape \color{blue} #1}}
\else
% without solutions
\newcommand{\soln}[1]{}
\fi


% Go to `Exam text starts from here' below to type in your questions. %

%%%%%%%%%%%%%%%%%%%%%%%%%%%%%%%%%%%%%%%%%%%%%%%%%%%%%%%%%%%%%%%%%
%%%%%%%%%%%%%%%%%%%%%%%%%%%%%%%%%%%%
% Ignore the following details     %
%%%%%%%%%%%%%%%%%%%%%%%%%%%%%%%%%%%%

\newcommand{\examyear}{2020}
\newcommand{\exammonth}{May -- June} 
\newcommand{\examdate}{??} %{Tuesday, 26th May 2018} 
\newcommand{\examtime}{??} %{09:30 -- 11:30} 
\newboolean{onebook}
\setboolean{onebook}{false} %set to true if all answers should be in one answer book

%%%%%%%%%%%%%%%%%%%%%%%%%%%%%%%%%%%%

\pretolerance=100000
\setlength{\topmargin}{-15mm}
\setlength{\textheight}{245mm}
\setlength{\textwidth}{178mm}
\setlength{\oddsidemargin}{-10mm}
\setlength{\evensidemargin}{-10mm}
\setlength{\marginparwidth}{1cm}
\setlength{\parskip}{1.1ex}
\setlength{\parindent}{0ex}
\renewcommand{\baselinestretch}{1.1}

\pagestyle{empty}

\newenvironment{Question}[1] 
 {\begin{itemize} \item[\large #1.~~]}{\end{itemize} \medskip}
 
\newcommand{\EndPage}{
	\vfill \coursenum ~ \coursename ~
	(\examyear) \hfill Page \thepage \newpage
	}

%\newcommand{\ICLOGO}{
%	\begin{minipage}{0.7\textwidth}
%  \includegraphics[height= 1.4cm]{imperial.pdf} 
%  \end{minipage}\hfill \parbox[r][1.4cm][t]{0.2\textwidth}{\hfill \coursenum} \par 
%	}
	
\newcommand{\BeginParts}{\begin{itemize}} 
\newcommand{\Part}[1]{\item [(#1)~~]} 
\newcommand{\EndParts}{\end{itemize}} 

%%%%
% This requires imperial.pdf in the graphics input path eg 
% h:/images/imperial.pdf 
%%%% 

\newcommand{\draft}{
	\begin{flushleft} 
	\begin{tabular}{ll}
  Module:   & \coursenum\\ Setter:   & \setter \\
  Checker:  & \checker \\  Editor:   & \editor \\
  External: & \external \\ Date:     & \today\\
  Version: & \version
  \end{tabular} 
  \end{flushleft} 
  \vfill \par
	}


\newcommand{\fpagedraft}{  % FOR DRAFT FRONT PAGE with SIGS
%	\ICLOGO
	\begin{center} 
	\draft
	BSc, MSci and MSc EXAMINATIONS (MATHEMATICS) \par 
	\exammonth~ \examyear 
	
	\medskip
	
	\large \coursenum \quad
	\coursename 
	\end{center}
	\medskip
	
	\textit{The following information must be completed:}
	
	\medskip
	
	\textbf{\suitability}
	
	\medskip
	
	\textbf{Category A marks: available for basic, routine material (excluding any mastery question)\\(40 percent = 32/80 for 4 questions):}\newline
	\CatA
	
	\medskip
	
	\textbf{Category B marks: Further 25 percent of marks (20/ 80 for  4 questions) for demonstration of a sound knowledge of a 
	good part of the material and the solution of straightforward problems and examples with reasonable accuracy (excluding mastery question):} \newline
	\CatB
	
	\medskip
	
	\textbf{Category C marks: the next 15 percent of the marks (= 12/80 for 4 questions)  for parts of questions at the high 2:1  or 1st class level:}\newline
	\CatC
	
	\medskip
	
	\textbf{Category D marks: Most challenging 20 percent (16/80 marks for 4 questions) of the paper (excluding mastery question):}\newline
	\CatD	
	
	
	
	 \vfill \par \normalsize
	 
	
	
	\textit{Signatures are required for the final version:}
		\sigs  \par \vfill 
	\copyright ~\examyear~ Imperial College London
	\hfill \coursenum \hfill Temporary cover page
	\newpage
	}


	
	
\newcommand{\sigs}{ % Signatures
	\vfill \par 
  \fbox{\begin{minipage}{0.98\textwidth}{~ \\[4mm]
  \hspace*{3mm} Setter's signature \hfill
  Checker's signature \hfill
  Editor's signature~~~~ \\[4mm]
  \hspace*{3mm} \dotfill \hfill \dotfill \hfill \dotfill ~~~~
  \\[2mm] ~} \end{minipage}}
	}
	


	
\newcommand{\fpage}{ % FRONT PAGE 
%	\ICLOGO
%  
\newlength{\myl}
\settowidth{\myl}{\sc Temporary front page}
\newlength{\myll}
\setlength{\myll}{\textwidth - \myl}
\begin{center}
 \large{\sc \rule[4pt]{0.45\myll}{0.5pt}~Temporary front page~\rule[4pt]{0.45\myll}{0.5pt}}
 \end{center}
 %
 \bigskip
  \begin{center}
  BSc, MSc and MSci EXAMINATIONS (MATHEMATICS) \par
  \exammonth~ \examyear \bigskip \par
  This paper is also taken for the relevant examination
  for the Associateship of the \\ Royal College of Science. 
	\end{center}
	\begin{center}
	\coursename
	\end{center}
	\renewcommand{\arraystretch}{1.3}
	\vfill \par
	\fbox{\parbox{\textwidth}{
	\begin{tabular}{l}
	Date: \examdate \\ 
	Time: \examtime \\
	Time Allowed: \examlength \\
	This paper has {\em \numQ}. \\
	\ifthenelse{\boolean{onebook}}
	{Candidates should use ONE main answer book.}
	{Candidates should start their solutions to each question in a new main answer book.}\\
	%Candidates should use ONE main answer book.\\
	Supplementary books may only be used after the relevant main book(s) are full. \\
	\ifthenelse{\boolean{tables}}{Statistical tables are provided.}{Statistical tables will not be provided.}\\
	\end{tabular}}}
	\renewcommand{\arraystretch}{1.5}
	\begin{itemize}
	\item DO NOT OPEN THIS PAPER UNTIL THE INVIGILATOR TELLS YOU TO.
	\item Affix one of the labels provided to each answer book that you use, 
	but DO NOT USE THE LABEL WITH YOUR NAME ON IT.
	\item	Credit will be given for all questions attempted.
	\item Each question carries equal weight.
	\item Calculators may not be used.
	\end{itemize}
	\vfill 
	\copyright ~\examyear~ Imperial College London
	\hfill \coursenum \hfill Page 1 of \pageref{ptotal}
	\newpage
	}
%%%%%%%%%%%%%%%%%%%%%%%%%%%%%%%%%%%%%%%%%%%%%%%%%%%%%%%%%%%%%%%%
%%%%%%%%%%%%%%%%%%%%%%%%%%%%%%%%%%%%%%%%%%%%%%%%%%%%%%%%%%%%%%%%
% Exam text starts from here
%%%%%%%%%%%%%%%%%%%%%%%%%%%%%%%%%%%%%%%%%%%%%%%%%%%%%%%%%%%%%%%%%
%%%%%%%%%%%%%%%%%%%%%%%%%%%%%%%%%%%%%%%%%%%%%%%%%%%%%%%%%%%%%%%%%

% use \begin{Question}{1} ... \end{Question} and
%     \BeginParts \Part{a} \EndParts
%
% manually put in \EndPage at the end of every page for the correct footers.

\begin{document}
\sffamily
\fpagedraft % this produces the signature page 
\setcounter{page}{1}
\fpage % this produces the front page with rubric

\begin{Question}{1}
This question is about the equation  
\begin{equation}
  \label{eq:adv-diff}
  c\phi + \MM{u}\cdot\nabla \phi-\epsilon\nabla^2 \phi = f \mbox{ on }\Omega, \quad
  \pp{\phi}{n} = 0 \mbox{ on }\partial\Omega,
\end{equation}
where:
\begin{itemize}
\item $\Omega$ is a $d$-dimensional polygonal domain with boundary $\partial\Omega$,
\item $c>0$,
\item $f$ is a known function,
\item $\MM{u}\in C^{1,\infty}(\Omega)^d$ is a known vector-valued
  function satisfying $\nabla\cdot\MM{u}=0$, and $\MM{u}\cdot\MM{n}=0$
  on $\partial\Omega$.
\item $|\MM{u}|_{\infty} = \max_{\MM{x}\in \Omega}|\MM{u}(\MM{x})| = C_0 < \infty$.
\end{itemize}

\BeginParts
\Part{a} Derive a weak formulation of this equation for a solution
$\phi\in H^1(\Omega)$ of the form
\begin{equation}
  a(q,\phi) = F(\phi), \quad \forall H^1(\Omega).
\end{equation}
\exammarks{5}
\soln{\unseen
  Multiplying by a test function $q$ and integrating by parts, we obtain
  \[
  \int_{\Omega} cq\phi -\nabla\cdot (\MM{u}q)\phi + \epsilon \nabla q\cdot\nabla\phi
  \diff x = \int_{\Omega} fq\diff x, \quad \forall q\in H^1(\Omega).
  \]
  Equal credit for the formulation without integrating by parts in the
  advection term (they are equivalent).
}
\Part{b} Obtain estimates for the continuity and coercivity constants
of $a(\cdot,\cdot)$.
\exammarks{10}
\soln{\unseen
  From Cauchy-Schwarz, 
  \begin{align*}
    |a(q,\phi)| & = \int_{\Omega} cq\phi -q\phi\nabla\cdot\MM{u} - \phi\MM{u}\cdot
    \nabla q + \epsilon \nabla q \cdot \nabla \phi \diff x, \\
    & \leq c\|q\|_{L^2(\Omega)}\|\phi\|_{L^2(\Omega)}
    + C_0\|\nabla q\|_{L^2(\Omega)}\|\phi\|_{L^2(\Omega)}
    + \epsilon \|\nabla q \|_{L^2(\Omega)}\|\nabla \phi\|_{L^2(\Omega)}, \\
    & \leq \left(c + C_0 + \epsilon\right)\|q\|_{H^1(\Omega)}
    \|\phi\|_{H^1(\Omega)}, 
  \end{align*}
  so the continuity constant is $c + C_0  + \epsilon$.\\
  To compute the coercivity constant, first note that the advection term
  is skew-symmetric, since
  \begin{align*}
    \int_{\Omega} \nabla\cdot (\MM{u}q)\phi \diff x & =
    \int_{\Omega} \underbrace{\nabla\cdot \MM{u}}_{=0} q\phi
    + (\MM{u}\cdot\nabla q) \phi\diff x, \\
    & = -\int_{\Omega} \nabla\cdot(\MM{u}\phi) q \diff x,
  \end{align*}
  after integrating by parts. Hence,
  \begin{align*}
    a(\phi,\phi) & = \int_{\Omega} c\phi^2 + \epsilon|\nabla \phi|^2
    \diff x, \\
    & > \min(c,\epsilon) \|\phi\|_{H^1(\Omega)},
  \end{align*}
  so the coercivity constant is $\min(c,\epsilon)$.
}
\Part{c} What happens to the $H^1$ norm of the error in the $P^1$ finite element approximation
of this problem as $\epsilon \to 0$? Justify your answer.
\exammarks{5}
\soln{\unseen From C\'ea's Lemma, we have
  \[
  \|\phi_h - \phi\|_{H^1(\Omega)} \leq
  \frac{c + C_0+\epsilon}{\epsilon}
  \|\phi\|^2_{H^1(\Omega)},
  \]
  (assuming that $c>\epsilon)$, which tends to infinity as $\epsilon
  \to 0$. Hence the error can be arbitrarily large in that limit.}
\EndParts
\end{Question}
\EndPage

\begin{Question}{2}
  \BeginParts
  \Part{a}
  For a ball $B$ in a triangle $K$, the averaged Taylor polynomial of
  a function $u \in H^{k}(K)$ of degree $k$ is defined by
  \begin{equation}
    Q_{k,B}u(x) = \frac{1}{|B|}\int_{B}\sum_{|\alpha|\leq k} D^\alpha u(y)
    \frac{(x-y)^\alpha}{\alpha!} \diff y.
  \end{equation}
  For $|\beta|\leq k$ show that
  \begin{equation}
    D^\beta Q_{k,B}u(\MM{x}) = Q_{k-|\beta|,B}D^\beta u(\MM{x}).
  \end{equation}
  \exammarks{8}
  \soln{\seen
    For continuous functions $u\in C^k(K)$, we have
    \[
    T_{k,y} u (x) = \sum_{|\alpha|\leq k}D^\alpha u(y)
    \frac{(x-y)^\alpha}{\alpha!}.
    \]
    Then,
    \begin{align*}
      D^\beta T_{k,y} u (x) &= D^\beta
      \sum_{|\alpha|\leq k}D^\alpha u(y)
      \frac{(x-y)^\alpha}{\alpha!}, \\
      &=
      \sum_{|\beta|\leq |\alpha|\leq k}D^\alpha u(y)
      \frac{(x-y)^{\alpha-\beta}}{(\alpha-\beta)!}, \\
      &=
      \sum_{|\alpha|\leq k-|\beta|}D^{\alpha+\beta} u(y)
      \frac{(x-y)^\alpha}{(\alpha)!}, \\
      &=
      \sum_{|\alpha|\leq k-|\beta|}D^{\alpha}D^{\beta} u(y)
    \frac{(x-y)^\alpha}{(\alpha)!}, \\
    = T_{k-|\beta|y} D^\beta u(x).
    \end{align*}
    Then,
    \begin{align*}
      D^\beta Q_{k,B} u(x) & =
      D^\beta \frac{1}{|\beta|}\int_B T_{k,y} u\diff x, \\
      & = \frac{1}{|\beta|}\int_B T_{k-|\beta|,y} D^\beta u\diff x, \\
      & = Q_{k-|\beta|,B} D^\beta u (x).
    \end{align*}
  }
  \Part{b} For the rest of the question we assume that $K$ has radius 1.
  Let $u\in H^{k+1}(K)$. 
  Assuming
  that, for $i\leq k$, 
  \begin{equation}
    \|Q_{i,B}u - u\|_{L^2(K)} \leq
    C|u|_{H^{k+1}(K)}, 
  \end{equation}
  show that
  \begin{equation}
    \|D^\beta (Q_{i,B}u - u) \|_{L^2(K)}
    \leq C|u|_{H^{k+1}(K)},
  \end{equation}
  for $|\beta| \leq i \leq k$.
  \exammarks{8}
  \soln{\seen From the previous result
    \[
      D^\beta (Q_{i,B}u - u)
        = Q_{i-|\beta|,B}D^\beta u - D^\beta u,
    \]
    so
    \[
      \|D^\beta (Q_{i,B}u - u)\|_{L^2(K)} = 
      \|Q_{i-|\beta|,B}D^\beta u - D^\beta u\|_{L^2(K)}
      \leq C|D^\beta u|_{H^{k-|\beta|+1}(K)}
      =C|u|_{H^{k+1}(K)}.
    \]
  }
  \Part{c} Using the property
  \begin{equation}
    \|I_Ku\|_{H^k(K)} \leq C_1\|u\|_{H^k(K)},
  \end{equation}
  for the nodal interpolation operator $I_K$ corresponding
  to a finite element $(K,\mathcal{P},\mathcal{N})$,
  show that
  \begin{equation}
    |I_Ku - u|_{H^k(K)} \leq C_2|u|_{H^{k+1}(K)},
  \end{equation}
  for some positive constant $C_2$, stating any assumptions
  you make about $(K,\mathcal{P},\mathcal{N})$.
  \exammarks{4}
  \soln{\seen
    \begin{align*}
      |I_ku - u|_{H^k(K)}
      &\leq |I_ku - Q_{k,B}u + Q_{k,B}u - u|_{H^k(K)}, \\
      &\leq |I_ku - Q_{k,B}u|_{H^k(K)} + |Q_{k,B}u - u|_{H^k(K)}, \\
      &= |I_k(u - Q_{k,B}u)|_{H^k(K)} + |Q_{k,B}u - u|_{H^k(K)}, \\
      &\leq |I_k(u - Q_{k,B}u)|_{H^k(K)} + |Q_{k,B}u - u|_{H^k(K)}, \\
      &\leq (C_1+1)|Q_{k,B}u - u|_{H^k(K)}, \\
      &\leq (C_1+1)C_2|u|_{H^{k+1}(K)},
    \end{align*}
    as required, where we used that $\mathcal{P}$ contains all polynomials
    of degree $k$ in the third line. In the last line we used the result
    of part (b) with $|\beta|=i=k$, with $C_2=CN$ where $N$ is the number of
    multi-indices $\beta$ with $|\beta|=k$.
    So, $C_2 \leq (C_1 + 1)C_2$.
  }
  \EndParts
\end{Question}
\EndPage

\begin{Question}{3}
  Consider the following triple $(K,\mathcal{P},\mathcal{N})$.
  \begin{itemize}
  \item $K$ is a triangle with vertices $z_1$, $z_2$, $z_3$.
  \item $\mathcal{P}$ are the polynomials of degree $\leq 3$.
  \item $\mathcal{N}$ are dual variables given by evaluations at
    $z_1 + (z_2-z_1)i/3 + (z_3-z_1)j/3$ for $0\leq i \leq j \leq 3$.
  \end{itemize}
  \BeginParts
  \Part{a} 
  Show that $\mathcal{N}$ determines $\mathcal{P}$.  \exammarks{10}
  \soln{ \similar Let $p\in \mathcal{P}$ be mapped to zero by all dual
    variables.  Restricted to $\Pi_1$, the line between $z_2$ and
    $z_1$, $p$ is a degree 3 polynomial vanishing at 4 places, so it
    is zero. Therefore $p=L_1(x)Q_1(x)$ where $L_1(x)$ is a non-degenerate
    linear function vanishing on $\Pi_1$. Working through the other two
    lines, we obtain that $p=cL_1(x)L_2(x)L_3(x)$, where none of the $L$
    functions vanish away from the three edges of the triangle. But $p$
    also vanishes at the centre of the triangle, so $c=0$ i.e. $p=0$
    as required.}
  \Part{b} Describe the geometric decomposition for this finite element,
  and explain why it is a $C^0$ decomposition.  \exammarks{10} \soln{
    \similar 
    We associate the point evaluation at vertices to their
    corresponding edges, point evaluation on edges away from vertices
    to their corresponding edges, and point evaluation at the centre
    to the triangle itself.  Being $C^0$ requires that when restricted
    to a vertex, the function can be reconstructed purely from the
    corresponding vertex node. This is clear because there is just one
    point value to reconstruct. Being $C^0$ also requires that when
    restricted to an edge, the function can be reconstructed purely
    from nodal variables assigned to the closure of the edge. Since we
    have 4 values along the edge including the two vertices, this
    completely determines the cubic function along that edge.  }
  \EndParts
\end{Question}
\EndPage

\begin{Question}{4}
  Consider the heat equation,
  \begin{equation}
    \label{eq:heat}
    \pp{T}{t} = \kappa \nabla^2 T,
  \end{equation}
  solved for a time-dependent function $T$ on a closed simply-connected domain $\Omega$,
  with boundary conditions $\pp{T}{n}=0$ on the boundary $\partial\Omega$.
  \BeginParts
  \Part{a} Given a $C^0$ finite element space, formulate a finite element
  discretisation of the heat equation \eqref{eq:heat}.
  \exammarks{5}
  \soln{\unseen
    The variational form is obtained by multiplying both sides by a test
    function $v$ and integrating by parts to obtain
    \[
      \langle v, T_t \rangle_{L^2} = -\kappa \langle \nabla v, \nabla T
      \rangle_{L^2}.
    \]
    The finite element discretisation is then find a time-dependent
    $T\in V_h$ such that
    \[
      \langle v, T_t \rangle_{L^2} = -\kappa \langle \nabla v, \nabla T
      \rangle_{L^2}, \quad \forall v \in V_h.
    \]
  }
  \Part{b} Show that the discretisation can be written in the form
  \begin{equation}
    M\dot{\MM{T}} = K\MM{T},
  \end{equation}
  where $\MM{T}$ is the vector of basis coefficients for $T$ in the
  finite element space $V_h$.
  \exammarks{5}
  \soln{\unseen 
    Introducing basis expansions
    \[
    v = \sum_i v_i \phi_i(x), \quad T = \sum_i T_i(t) \phi(x),
    \]
    we get
    \[
      \sum_i v_i \sum_j\left(\langle \phi_i, \phi_j \rangle_{L^2}\dot{T}_j
      + \kappa \langle \nabla \phi_i, \nabla \phi_j \rangle_{L^2} T_j\right)
      = 0,
      \]
    but this equation must hold for arbitrary basis coefficients, so
    \[
    \sum_j\left(\underbrace{\langle \phi_i, \phi_j \rangle_{L^2}}_{=M_{ij}}\dot{T}_j
      + \kappa\underbrace{\langle \nabla \phi_i, \nabla \phi_j \rangle_{L^2}}_{=K_{ij}} T_j\right)
      = 0,
      \]
      as required.
  }
      \Part{c} Quoting results from lectures, show that
      \begin{equation}
        \frac{d}{dt} \int_{\Omega} T^2 \diff x \leq -C
        \int_\Omega T^2 \diff x,
      \end{equation}
      providing an upper bound for the decay rate $C$.
      \exammarks{5}
      \soln{\unseen For Dirichlet boundary conditions we have the result
        \[
        \int_\Omega T^2 \diff x \leq C_p\int_\Omega |\nabla T|^2 \diff x.
        \]
        Then, taking $v=T$ in the variational form,
        \begin{align*}
          \frac{d}{dt} \int_\Omega T^2 \diff x & = 2\int_\Omega TT_t \diff x, \\
          & = -2\kappa \int_\Omega |\nabla T|^2\diff x, \\
          & \leq -\frac{2\kappa}{C_p} \int_\Omega T^2 \diff x.
        \end{align*}
      }
      \Part{d} Explain why this means that the decay rate for the
      finite element discretisation is larger than or equal to the
      decay rate for the unapproximated equation.  \exammarks{5}
      \soln{\unseen The Poincar\'e constant for $\mathring{H}^1$ is
        \[
        C^*_p = \sup_{T \in \mathring{H}^1(\Omega)}\frac
        {\int_\Omega T^2 \diff x}{\int_\Omega |\nabla T|^2\diff x}.
        \]
        The Poincar\'e constant for $\mathring{V}_h \subset \mathring{H}^1$ is
        \[
        C^h_p = \sup_{T \in \mathring{V}_h}\frac{\int_\Omega T^2 \diff x}
        {\int_\Omega |\nabla T|^2\diff x}
        \leq 
        \sup_{T \in \mathring{H}^1}
        \frac{\int_\Omega |T|^2\diff x}{\int_\Omega |\nabla T|^2 \diff x}
        = C^*_p,
        \]
        so $C^h_p \leq C^*_p$.
      }
  \EndParts
\end{Question}
\EndPage

\begin{Question}{5}
  This question is based upon the Mastery material ``From Functional Analysis to
  Iterative Methods'' by RC Kirby.\\
  Consider the partial differential equation
  \begin{equation}
    -\nabla\cdot (\gamma(x) \nabla u) = f, 
  \end{equation}
  on $\Omega$, with boundary conditions $u=0$ on $\partial\Omega$,
  where $f$ is a known function with $\|f\|_{L^2(\Omega)}<\infty$,
  and $\gamma$ is a known function with $c_1\leq \gamma \leq c_2$
  for $c_1>0$, $c_2<\infty$. \\
  \BeginParts
  \Part{a}
  Briefly formulate a finite element discretisation for this problem
  using linear continuous finite elements,
  and explain how the coercivity
  and continuity constants of the variational problem depend on $c_1$ and $c_2$.
  Give details on the function spaces involved and norms involved.
  \exammarks{6}
   \soln{\similar
    The weak form is
    \[
    \int_\Omega \gamma \nabla v\cdot \nabla u \diff x
    = \int_\Omega vf\diff x,
    \]
    with the variational problem defined on $\mathring{H}^1(\Omega)$,
    the subspace of $H^1(\Omega)$ with traces satisfying the zero
    Dirichlet boundary condition. Then the finite element
    discretisation has the same weak form with $\mathring{V}\subset
    \mathring{H}^1(\Omega)$.

    The bilinear form is continuous since
    \[
    \int_\Omega \gamma \nabla v \cdot \nabla u \diff x
    \leq b \|u\|_{H^1(\Omega)}\|v\|_{H^1(\Omega)},
    \]
    so  the continuity constant is $b$, 
    and coercive since
    \[
    \int_\Omega \gamma |\nabla u|^2 \diff x
    \geq a|u|^2_{H^1(\Omega)} \geq a(1+C)\|u\|^2_{H^1(\Omega)},
    \]
    where $C$ is constant for the Poincar\`e inequality, so the
    coercivity constant is $a(1+C)$.
   }
   
   \Part{b} A bilinear form $a$ on a finite element space $V_h$ defines
   an operator $A_h:V_h\to V_h'$ into the dual space given by
   \begin{equation}
     (A_h f)[u] = a(f,u), \quad \forall f,u \in V_h.
   \end{equation}
   In the notation of the paper, the operator
   $\mathcal{I}_h:\mathbb{R}^{\dim V_h} \to V_h$ maps a vector to the
   function in $V_h$ with the vector entries as basis coefficients in
   the nodal basis expansion. The operator
   $\mathcal{I}'_h:\mathbb{R}^{\dim V_h} \to V_h'$ maps vectors to
   linear functionals $F \in V_h'$ given by
   \begin{equation}
     (\mathcal{I}'_h\MM{f})[u] = \MM{f}^T(\mathcal{I}^{-1}_hu),
     \quad \forall u \in V_h.
   \end{equation}
   \BeginParts
   \Part{i}
   Show that
   \begin{equation}
     A_h u = \mathcal{I}'_h(A\MM{u}), \quad \forall u \in V_h.
   \end{equation}
   where $A$ is the matrix corresponding to $A_h$ and
   $\MM{u}$ is the vector of basis coefficients of $u$.
   \exammarks{4}
   \soln{\seen
     Let $v=\mathcal{I}_h\MM{v}$. Then
     \begin{align*}
       \langle A_hu, v\rangle & = a(u,v), \\
       &= a\left( \sum_i u_i \phi_i, \sum_j v_j\phi_j\right), \\
       &= \sum_{i,j} u_iv_j a(\phi_i,\phi_j), \\
       &= \sum_{i,j} u_iv_j A_{ij}, \\
       &= (A\MM{u})^T\MM{v}, \\
       &= (A\MM{u})^T(\mathcal{I}_h^{-1}\mathcal{I}_h\MM{v}), \\
       &= \langle \mathcal{I}'_hA\MM{u},v \rangle, \quad
       \forall v \in V_h,
     \end{align*}
     as required.
   }
   \Part{ii} Hence show that
   \begin{equation}
     A = (\mathcal{I}_h')^{-1}A_h \mathcal{I}_h.
   \end{equation}
   \exammarks{3}
   \soln{\seen
     We have $\mathcal{I}_h\MM{u}=u$, hence
     \[
     A\MM{u} = (\mathcal{I}')_h^{-1}A_hu = (\mathcal{I}')_h^{-1}A_h\mathcal{I}_h\MM{u},
     \quad \forall \MM{u} \in V_h,
     \]
     as required.
   }
   \EndParts

   \Part{c} Now consider a second bilinear form
   \begin{equation}
     b_h(u,v) = \int_{\Omega} uv + \nabla u \cdot \nabla v \diff x,
   \end{equation}
   with corresponding matrix $B$, and operator $B_h:V_h \to V_h'$.
   \BeginParts
   \Part{i} Show that
   \begin{equation}
     B^{-1}A = \mathcal{I}_h^{-1} B_h^{-1} A_h \mathcal{I}_h.
   \end{equation}
   \exammarks{4}
   \soln{\seen
     \begin{align*}
       B^{-1}A & = ((\mathcal{I}_h')^{-1}B_h \mathcal{I}_h)^{-1}
       (\mathcal{I}_h')^{-1}A_h \mathcal{I}_h, \\
       & = (\mathcal{I}_h)^{-1}B_h^{-1} \mathcal{I}_h'
       (\mathcal{I}_h')^{-1}A_h \mathcal{I}_h, \\
       & = (\mathcal{I}_h)^{-1}B_h^{-1} 
       A_h \mathcal{I}_h, 
     \end{align*}
     as required.
   }
   \Part{ii} Explain why $B_h^{-1}A_h$ has the same eigenvalues as $B^{-1}A$.
   \exammarks{3} \soln{\seen $B^{-1}A$ is similar to $B_h^{-1}A_h$ and
     so they have the same eigenvalues.
   }  \EndParts \EndParts
   \end{Question}

\label{ptotal}
\EndPage
\end{document}
