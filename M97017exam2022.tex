
\documentclass[12pt]{article}

\usepackage{graphicx,amssymb,amsmath,nicefrac,ifthen,calc}
\usepackage[T1]{fontenc}
\usepackage{lmodern}

%%%%%%%%%%%%%%%%%%%%%%%%%%%%%%%%%%%
%  Fill in the following details  %
%%%%%%%%%%%%%%%%%%%%%%%%%%%%%%%%%%%

\newcommand{\coursenum}{MATH97095} %{e.g: MATH50001, or: MATH96057/MATH97006/MATH97171 for multiple year courses}
\newcommand{\coursename}{Finite Elements} %{e.g: Mathematical Logic} 
\newcommand{\setter}{Cotter}     % Insert surname(s) of exam setter(s) 
\newcommand{\checker}{Ham} % Insert surname of exam checker 
\newcommand{\editor}{Wu} % Insert surname of exam editor
\newcommand{\external}{external} % Insert surname of external examiner
\newboolean{tables}
\setboolean{tables}{false} %set to true if statistical tables are provided and attach require tables

% Choose an option for time allowed and number of questions %

\newcommand{\examlength}%Uncomment one of the following 4 options, or adjust.
%{2~Hours}
%{2.5~Hours}
{2 Hours for MATH96 paper; 2.5 Hours for MATH97 papers}
%{3~Hours} %For 2-term modules

\newcommand{\numQ}%Uncomment one of the following 4 options, or adjust.
%{4~Questions}
%{5~Questions}
{4~Questions (MATH96 version); 5~Questions (MATH97~versions)}
%{6~Questions} %For 2-term modules

\newcommand{\version}% Uncomment one of the following, or adjust
{Draft Version}
%{Version for External Examiner}
%{Final version}


\newcommand{\suitability}% Uncomment one of the options below
{Is the paper suitable for resitting students from previous years:
  Yes, but only from 2020/21, because in years before that the mastery
  content was different.
}



%%%%%%%%%%%%%%%%%%%%%%%
%The following details about the difficulty level of the questions must be completed; also include this information on your mark-scheme.
%%%%%%%%%%%%%%%%%%%%%%%

\newcommand{\CatA}% Indicate, with marks, the most straightforward 40 percent of the marks available in the non-mastery questions
           {
1a:7, 1b:7, 1c:6, 2a:6, 2b:6 [Total 32]
}
\newcommand{\CatB}% Indicate the next most challenging 25 percent of the marks in the non-mastery questions
           {
4a:6, 4b:6, 2c:8 [Total 20]
}
\newcommand{\CatC}%Indicate the next most challenging 15 percent of the marks in the non-mastery questions
           {
             3a:6, 3c:6 [Total 12]

}
\newcommand{\CatD}% Indicate the most challenging 20 percent of marks on the paper in the non-mastery questions
           {
             4c:8, 3b:8 [Total 16]
           }



%%%%%%%%%%%%%%%%%%%%%%%%%%%%%%%%%%%%%
% Insert your macros %
%%%%%%%%%%%%%%%%%%%%%%%%%%%%%%%%%%%%%

\newcommand{\pp}[2]{\frac{\partial #1}{\partial #2}} 
\newcommand{\dede}[2]{\frac{\delta #1}{\delta #2}}
\newcommand{\dd}[2]{\frac{\diff#1}{\diff#2}}
\newcommand{\dt}[1]{\diff\!#1}
\def\MM#1{\boldsymbol{#1}}
\DeclareMathOperator{\diff}{d}
\DeclareMathOperator{\div0}{div}

% Go to `Exam text starts from here' below to type in your questions. %

%%%%%%%%%%%%%%%%%%%%%%%%%%%%%%%%%%%%%%%%%%%%%%%%%%%%%%%%%%%%%%%%%
%%%%%%%%%%%%%%%%%%%%%%%%%%%%%%%%%%%%
% Ignore the following details     %
%%%%%%%%%%%%%%%%%%%%%%%%%%%%%%%%%%%%

\newcommand{\examyear}{2021}
\newcommand{\exammonth}{May -- June} 
\newcommand{\examdate}{Wednesday, 5th May 2021} %{Tuesday, 26th May 2021} 
\newcommand{\examtime}{09:30 -- 11:30} 
\newboolean{onebook}
\setboolean{onebook}{true} %set to true if all answers should be in one answer book

%%%%%%%%%%%%%%%%%%%%%%%%%%%%%%%%%%%%
% setup a counter to count up marks in a question
\newcounter{tmarks}
\setcounter{tmarks}{0}

% Define the command for formatting the marks available for the question


\newcommand{\qmarks}[1]{\addtocounter{tmarks}{#1}\ifthenelse{\equal{#1}{1}}{\parbox{0.1in}{\ }\hfill{(#1 mark)}}{\parbox{0.1in}{\ }\hfill{(#1 marks)}}}

\newcommand{\qqmarks}[1]{\addtocounter{tmarks}{#1}\ifthenelse{\equal{#1}{1}}{\newline\parbox{0.1in}{\ }\hfill{(#1 mark)}}{\newline\parbox{0.1in}{\ }\hfill{(#1 marks)}}}

% define a total marks command for the end of the question
\newcommand{\totmarks}{\ifthenelse{\equal{\value{tmarks}}{1}}{\parbox{0.1in}{\ }\hfill{(Total: \arabic{tmarks} mark)}}{\parbox{0.1in}{\ }\hfill{(Total: \arabic{tmarks} marks)}}}





\pretolerance=100000
\setlength{\topmargin}{-15mm}
\setlength{\textheight}{245mm}
\setlength{\textwidth}{178mm}
\setlength{\oddsidemargin}{-10mm}
\setlength{\evensidemargin}{-10mm}
\setlength{\marginparwidth}{1cm}
\setlength{\parskip}{1.1ex}
\setlength{\parindent}{0ex}
\renewcommand{\baselinestretch}{1.1}

\pagestyle{empty}

\newenvironment{Question}[1] 
 {\begin{itemize} \item[\large #1.~~]}{\end{itemize}\totmarks\setcounter{tmarks}{0} \medskip}
 
\newcommand{\EndPage}{
	\vfill \coursenum ~ \coursename ~
	(\examyear) \hfill Page \thepage \newpage
	}

%\newcommand{\ICLOGO}{
%	\begin{minipage}{0.7\textwidth}
%  \includegraphics[height= 1.4cm]{imperial.pdf} 
%  \end{minipage}\hfill \parbox[r][1.4cm][t]{0.2\textwidth}{\hfill \coursenum} \par 
%	}
	
\newcommand{\BeginParts}{\begin{itemize}} 
\newcommand{\Part}[1]{\item [(#1)~~]} 
\newcommand{\EndParts}{\end{itemize}} 

%%%%
% This requires imperial.pdf in the graphics input path eg 
% h:/images/imperial.pdf 
%%%% 

\newcommand{\draft}{
	\begin{flushleft} 
	\begin{tabular}{ll}
  Module:   & \coursenum\\ Setter:   & \setter \\
  Checker:  & \checker \\  Editor:   & \editor \\
  External: & \external \\ Date:     & \today\\
  Version: & \version
  \end{tabular} 
  \end{flushleft} 
  \vfill \par
	}


\newcommand{\fpagedraft}{  % FOR DRAFT FRONT PAGE with SIGS
%	\ICLOGO
	\begin{center} 
	\draft
	BSc, MSci and MSc EXAMINATIONS (MATHEMATICS) \par 
	\exammonth~ \examyear 
	
	\medskip
	
	\large \coursenum \quad
	\coursename 
	\end{center}
	\medskip
	
	\textit{The following information must be completed:}
	
	\medskip
	
	\textbf{\suitability}
	
	\medskip
	
	\textbf{Category A marks: available for basic, routine material (excluding any mastery question)\\(40 percent = 32/80 for 4 questions):}\newline
	\CatA
	
	\medskip
	
	\textbf{Category B marks: Further 25 percent of marks (20/ 80 for  4 questions) for demonstration of a sound knowledge of a 
	good part of the material and the solution of straightforward problems and examples with reasonable accuracy (excluding mastery question):} \newline
	\CatB
	
	\medskip
	
	\textbf{Category C marks: the next 15 percent of the marks (= 12/80 for 4 questions)  for parts of questions at the high 2:1  or 1st class level (excluding mastery question):}\newline
	\CatC
	
	\medskip
	
	\textbf{Category D marks: Most challenging 20 percent (16/80 marks for 4 questions) of the paper (excluding mastery question):}\newline
	\CatD	
	
	
	
	 \vfill \par \normalsize
	 
	
	
	\textit{Signatures are required for the final version:}
		\sigs  \par \vfill 
	\copyright ~\examyear~ Imperial College London
	\hfill \coursenum \hfill Temporary cover page
	\newpage
	}


	
	
\newcommand{\sigs}{ % Signatures
	\vfill \par 
  \fbox{\begin{minipage}{0.98\textwidth}{~ \\[4mm]
  \hspace*{3mm} Setter's signature \hfill
  Checker's signature \hfill
  Editor's signature~~~~ \\[4mm]
  \hspace*{3mm} \dotfill \hfill \dotfill \hfill \dotfill ~~~~
  \\[2mm] ~} \end{minipage}}
	}
	


	
\newcommand{\fpage}{ % FRONT PAGE 
%	\ICLOGO
%  
\newlength{\myl}
\settowidth{\myl}{\sc Temporary front page}
\newlength{\myll}
\setlength{\myll}{\textwidth - \myl}
\begin{center}
 \large{\sc \rule[4pt]{0.45\myll}{0.5pt}~Temporary front page~\rule[4pt]{0.45\myll}{0.5pt}}
 \end{center}
 %
 \bigskip
  \begin{center}
  BSc, MSc and MSci EXAMINATIONS (MATHEMATICS) \par
  \exammonth~ \examyear \bigskip \par
  This paper is also taken for the relevant examination
  for the Associateship of the \\ Royal College of Science. 
	\end{center}
	\begin{center}
	\coursename
	\end{center}
	\renewcommand{\arraystretch}{1.3}
	\vfill \par
	\fbox{\parbox{\textwidth}{
	\begin{tabular}{l}
	Date: \examdate \\ 
	Time: \examtime \\
	Time Allowed: \examlength \\
	This paper has {\em \numQ}. \\
	%\ifthenelse{\boolean{onebook}}
	%{Candidates should use ONE main answer book.}
	%{Candidates should start their solutions to each question in a new main answer book.}\\
	%Candidates should use ONE main answer book.\\
	%Supplementary books may only be used after the relevant main book(s) are full. \\
	\ifthenelse{\boolean{tables}}{Statistical tables are provided.}{Statistical tables will not be provided.}\\
	\end{tabular}}}
	\renewcommand{\arraystretch}{1.5}
	\begin{itemize}
	%\item DO NOT OPEN THIS PAPER UNTIL THE INVIGILATOR TELLS YOU TO.
	%\item Affix one of the labels provided to each answer book that you use, 
	%but DO NOT USE THE LABEL WITH YOUR NAME ON IT.
	\item	Credit will be given for all questions attempted.
	\item Each question carries equal weight.
	%\item Calculators may not be used.
	\end{itemize}
	\vfill 
	\copyright ~\examyear~ Imperial College London
	\hfill \coursenum \hfill Page 1 of \pageref{ptotal}
	\newpage
	}
%%%%%%%%%%%%%%%%%%%%%%%%%%%%%%%%%%%%%%%%%%%%%%%%%%%%%%%%%%%%%%%%
%%%%%%%%%%%%%%%%%%%%%%%%%%%%%%%%%%%%%%%%%%%%%%%%%%%%%%%%%%%%%%%%
% Exam text starts from here
%%%%%%%%%%%%%%%%%%%%%%%%%%%%%%%%%%%%%%%%%%%%%%%%%%%%%%%%%%%%%%%%%
%%%%%%%%%%%%%%%%%%%%%%%%%%%%%%%%%%%%%%%%%%%%%%%%%%%%%%%%%%%%%%%%%

% Use \begin{Question}{1} ... \end{Question} and
%     \BeginParts \Part{a} \EndParts
%
% Avoid splitting a question across two pages
%
% Insert number of marks for each part or sub-part using \qmarks{}. Use \qqmarks{} if this results in a bad line break.
%
% Manually put in \EndPage at the end of every page for the correct footers.

\begin{document}
\sffamily
\fpagedraft % this produces the signature page 
\setcounter{page}{1}
\fpage % this produces the front page with rubric


\begin{Question}{1}
  Consider the following finite element
    \begin{itemize}
    \item $K$ is a triangle,
    \item $P$ is the space of polynomials of degree $\leq 2$,
    \item $N$ is the set of six nodal variables given by evaluation
      at the vertices and edge centres of $K$.
    \end{itemize}
    \BeginParts
  \Part{a} Show that $N$ determines $P$.
  \qmarks{7}
  \Part{b} Give a $C^0$ geometric decomposition of this finite element,
  showing that it is $C^0$.
  \qmarks{7}
  \Part{c} Show that finite element spaces built from this element are
  not necessarily $C^1$.
  \qmarks{6}
  \EndParts
\end{Question}

\EndPage

\begin{Question}{2}
  \BeginParts
\Part{a}
Write a $C^0$ finite element variational problem for the following equation,
\begin{equation}
 \epsilon u -\nabla^2 u = \exp(xy), \quad \frac{\partial u}{\partial n}=0 \mbox{ on } \partial\Omega,
\end{equation}
where $\Omega=\{x,y:0\leq x\leq 1,\, 0\leq y\leq 1\}$ with boundary
$\partial\Omega$, and $0<\epsilon<1$.
\qqmarks{6}
\Part{b}
Show that the bilinear form for the variational problem is continuous
and coercive, and give bounds for the continuity and coercivity
constants $M$ and $\gamma$ for this problem.\\ (You may make use of
the inequality $ab \leq \frac{a^2}{2}+\frac{b^2}{2}$.)  \qqmarks{6}
\Part{c}
Assuming Ce\'a's Lemma and standard interpolation error estimates,
derive an error bound for the $H^1$ error $\|u-u_h\|$ where $u_h$ is
the solution obtained by a linear Lagrange finite element approximation with maximum
mesh size $h$, and $u$ is the exact solution. What is happening to
this error bound when $\epsilon$ is very small?  \qqmarks{8} \EndParts
\end{Question}

 \EndPage
 \begin{Question}{3}
     \begin{figure}
    \centerline{\includegraphics[width=6cm]{BDM1_DOFs}}
    \caption{\label{fig:BDM1} Nodal variables diagram for Question 4.}
  \end{figure}
   In this question we consider the following finite element.
   \begin{itemize}
   \item $K$ is a triangle.
   \item $P$ are vector-valued linear functions (i.e. the $x$- and $y$-
     components of the function are both polynomials of degree $\leq 1$).
   \item The six nodal variables are the components of the function
     tangential to the edges at the locations indicated by the arrows
     in Figure \ref{fig:BDM1}.
   \end{itemize}
     \BeginParts
   \Part{a} Describe how this element can be used to construct a finite
     element space $V$ where the functions are continuous in the tangential
     component across each edge. Show that the finite element space
     does indeed have this property.
     \qmarks{6}
     \Part{b} Consider the quadratic Lagrange finite element space $P_2$.
   Show that $\phi \in P_2 \implies \nabla \phi \in V$.
   \qmarks{8}
   \Part{c} Provide a formula for the weak curl $\nabla^\perp\cdot u=-\partial u_1/\partial y + \partial u_2/\partial x$
   for a function $u=(u_1,u_2)\in V$, and show that it is indeed the
   weak curl.  \qmarks{6} \EndParts
\end{Question}
\EndPage
 \begin{Question}{4}
  Let $\Omega$ be a convex polygonal domain. Assume that you have a fast and
  efficient code for solving the variational problem: find $u \in V$
  such that
  \begin{equation}
    \int_\Omega uv + \nabla u \cdot \nabla v \diff x = F[v], \quad
    \forall v \in V,
  \end{equation}
  for arbitrary linear functionals $F[v]$, where $V$ is a $C^0$ finite
  element space. However, you want to solve a different variational
  problem: find $u \in V$ such that
  \begin{equation}
    \label{eq:actual}
    \int_\Omega a(x) uv + b(x)\nabla u \cdot \nabla v \diff x = G[v], \quad
    \forall v \in V,
  \end{equation}
  where $a(x)$ and $b(x)$ are some known functions that satisfy
  $0<\alpha< a(x) < \beta < \infty$, $0<\alpha< b(x) < \beta <
  \infty$, for all $x\in \Omega$. One possible approach is to apply
  the following iterative scheme,
  \begin{equation}
    \int_\Omega u^{k+1}v + \nabla u^{k+1} \cdot \nabla v \diff x =
    F_k[v],
    \label{eq:it}
  \end{equation}
  where
  \begin{equation}
    F_k[v] = \int_\Omega u^{k}v + \nabla u^{k} \cdot \nabla v \diff x
    + \mu\left(
    G[v] -
    \int_\Omega a(x) u^kv + b(x)\nabla u^k \cdot \nabla v \diff x\right)
    , \quad
    \forall v \in V,
  \end{equation}
  where $\mu>0$, for an iterative sequence $u^0,u^1,u^2,\ldots$
  of guesses at the solution.  To implement this, we choose an initial
  guess $u^0$, and then iteratively generate the sequence by solving
  \eqref{eq:it} for $u^{k+1}$ given $u^k$ (which enables us to
  construct $F_k[v]$).  \BeginParts
  \Part{a} Show that if the sequence converges to a limit $u_k \to u^*$ as
  $k\to \infty$, then $u^*$ solves Equation \eqref{eq:actual}.
  \qmarks{6}
  \Part{b}
  Defining the error $\epsilon^k = u - u^k$, where $u$ solves
  \eqref{eq:actual}, derive a variational problem that relates
  $\epsilon^{k+1}$ to $\epsilon^k$ (without explicitly involving
  $u^{k+1}$ or $u^k$).  \qmarks{6}
  \Part{c} 
  Find a
  value of $\mu$ such that
  $\|\epsilon^{k+1}\|_{H^1}<\|\epsilon^k\|_{H^1}$, concluding that
  the iterative procedure converges.
  You may make use of the stability bound from Lax-Milgram, i.e.
  the solution $u$ to a variational problem satisfies
  \begin{equation}
    \|u\|_{H^1} \leq \frac{1}{\gamma}\|F\|_{(H^1)^*},
  \end{equation}
  where $\gamma$ is the coercivity constant of the bilinear form and
  $F$ is the linear form appearing on the right hand side.
  \qmarks{8}
  \EndParts
\end{Question}

%% If you have a Mastery Question, then this should be included as an additional question 
%% on a new page. Otherwise, comment out the following three lines:
\EndPage
\begin{Question}{5}
 \BeginParts
 \Part{a} Let $V$ and $Q$ be Hilbert spaces. Let $b:V\times Q \to \mathbb{R}$ be a bilinear form.
 We define the operator $B:V\to Q'$ as follows. For each $v\in V$,
 $Bv$ is an element of $Q'$, defined by
 \begin{equation}
   (Bv)[p] = b(v,p), \, \forall p \in Q.
 \end{equation}
 For an operator $T:X\to Y'$, we define the transpose operator
 $X^*:Y\to X'$ as
 \begin{equation}
   (T^*y)[x] = (Tx)[y], \quad \forall x\in X,\,y\in Y.
 \end{equation}
 Use these definitions to derive a formula for $B^*$.
 \qmarks{6}
 \Part{b} Assuming the inf-sup condition
 \begin{equation}
   \inf_{0\neq q\in Q}\sup_{0\neq v\in V}
   \frac{b(v,q)}{\|v\|_V\|q\|_Q} \geq \beta,
 \end{equation}
 for some $\beta>0$, show that $B^*$ is injective.
 \qmarks{7}
 \Part{c}
 Let
 \begin{equation}
   b(u,p) = \int_{\Omega} p \nabla\cdot u \diff x,
 \end{equation}
 for some chosen problem domain $\Omega$ such that
 $b$ satisfies the inf-sup condition for some given finite
 element spaces $V_h$ and $Q_h$. We define the ``weak gradient''
 operator $\tilde{\nabla}:Q_h\to V_h$ such that
 \begin{equation}
   \int_\Omega w\cdot \tilde{\nabla}p\diff x
   = \int_\Omega p\nabla\cdot u \diff x.
 \end{equation}
 What does the inf-sup condition imply about the operator
 $\tilde{\nabla}$?
 \qmarks{7}
 \EndParts
 \end{Question}


\label{ptotal}
\EndPage
\end{document}
