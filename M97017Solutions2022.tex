\documentclass[11pt]{article}
\usepackage{graphicx, amssymb}
\usepackage{amsfonts}
\usepackage{epsfig}
\usepackage{amssymb,amsmath,bm}
%\usepackage{subfig}
\usepackage{amsthm}
\usepackage{mathrsfs,mathtools}


%%%%%%%%%%%%%%%%%%%%%%%%%%%%%%%%%%%
%  Fill in these details
\newcommand{\coursenum}{MATH97095} %{M2S1}
\newcommand{\coursename}{MATH97095 (Solutions)} 
\renewcommand{\pagetotal}{\pageref{pagetot}}     % adjust after viewing draft 
\newcommand{\examyear}{2022}%Please update 
\newcommand{\exammonth}{May} %Please update

%%%%%%%%%%%%%%%%%%%%%%%%%%%%%%%%

\pretolerance=100000
 \setlength{\topmargin}{-3mm}
 \setlength{\textheight}{250mm}


%%%%%%%%%%%%%%%%%%%%%%%%%%%%%%%%%%%%%%%% 
%Added the following
\setlength{\footskip}{0mm}
\setlength{\headsep}{0mm}
\setlength{\headheight}{0mm}



%%%%%%%%%%%%%%%%%%%%%%%%%%%%%%%%

\setlength{\textwidth}{150mm}
\addtolength{\oddsidemargin}{-20mm}
\addtolength{\evensidemargin}{40mm}
\setlength{\marginparwidth}{2cm}
\setlength{\parskip}{1.1ex}
\setlength{\parindent}{0ex}
\renewcommand{\baselinestretch}{1.1}

% seen, unseen and similar seen commands; add in your own as appropriate
\newcommand{\marb}[1]{\marginpar{~~\fbox{\,\mathstrut #1\,}}}  
\newcommand{\marbup}[1]{\vspace*{-4mm} \par \marb{#1}} 
\newcommand{\seen}{\marbup{{\small seen $\Downarrow$}}} 
\newcommand{\unseen}{\marbup{{\small unseen $\Downarrow$}}} 
\newcommand{\simseen}{\marbup{{\small sim. seen $\Downarrow$}}} 
\newcommand{\seensim}{\marbup{{\small seen/sim.seen $\Downarrow$}}} 
\newcommand{\partseen}{\marbup{{\small part seen $\Downarrow$}}} 
\newcommand{\methodseen}{\marbup{{\small meth seen $\Downarrow$}}} 

\pagestyle{empty}
\newcommand{\vect}[1]{\mbox{\boldmath{$#1$}}}

\newcommand{\pp}[2]{\frac{\partial #1}{\partial #2}} 
\newcommand{\dede}[2]{\frac{\delta #1}{\delta #2}}
\newcommand{\dd}[2]{\frac{\diff#1}{\diff#2}}
\newcommand{\dt}[1]{\diff\!#1}
\def\MM#1{\boldsymbol{#1}}
\DeclareMathOperator{\diff}{d}


% setup a counters to count up marks in a question
\newcounter{count_marks}
\setcounter{count_marks}{0}
\newcounter{count_amarks}
\setcounter{count_amarks}{0}
\newcounter{count_bmarks}
\setcounter{count_bmarks}{0}
\newcounter{count_cmarks}
\setcounter{count_cmarks}{0}
\newcounter{count_dmarks}
\setcounter{count_dmarks}{0}
\newcounter{count_mmarks} %Mastery marks
\setcounter{count_mmarks}{0} %Mastery marks
% Define the command for formatting the marks available for the question
\newcommand{\amarks}[1]{\addtocounter{count_marks}{#1} \addtocounter{count_amarks}{#1}\marginpar{~~\fbox{\,\mathstrut #1, A}}}  

\newcommand{\bmarks}[1]{\addtocounter{count_marks}{#1} \addtocounter{count_bmarks}{#1}\marginpar{~~\fbox{\,\mathstrut #1, B}}}  

\newcommand{\cmarks}[1]{\addtocounter{count_marks}{#1} \addtocounter{count_cmarks}{#1}\marginpar{~~\fbox{\,\mathstrut #1, C}}}  

\newcommand{\dmarks}[1]{\addtocounter{count_marks}{#1} \addtocounter{count_dmarks}{#1}\marginpar{~~\fbox{\,\mathstrut #1, D}}}  

\newcommand{\mmarks}[1]{\addtocounter{count_marks}{#1} \addtocounter{count_mmarks}{#1}\marginpar{~~\fbox{\,\mathstrut #1, M}}}  %Mastery marks


% define a total marks command for the end of the question
\newcommand{\totamarks}{\hfill{Total A marks: \arabic{count_amarks} of 
%24 %3 questions
%48 %6 questions
32 %4 questions
 marks}}
\newcommand{\totbmarks}{\hfill{Total B marks: \arabic{count_bmarks} of 
%15 %3 questions
% 30 % 6 questions
20 %4 questions
%Please adjust the number accordingly
marks}}
\newcommand{\totcmarks}{\hfill{Total C marks: \arabic{count_cmarks} of 
%9 %3 questions
% 18 %6 questions
12 %4 questions
%Please adjust the number accordingly
marks}}
\newcommand{\totdmarks}{\hfill{Total D marks: \arabic{count_dmarks} of 
%12 % 3 questions
% 24 % 6 questions
16 % 4 questions
%Please adjust the number accordingly
marks}}

\newcommand{\totmarks}{\hfill{Total marks: \arabic{count_marks} of 
%60 % 3 questions
%120 % 6 questions
80 %4 questions
%Please adjust the number accordingly
marks}}

\newcommand{\totMmarks}{\hfill{Total Mastery marks: \arabic{count_mmarks} of 
20  marks}}

\newenvironment{Question}[1] 
 {\begin{itemize} \item[\large #1.~~]}{\end{itemize}\vfill
} 
%\makeatletter
\newenvironment{Part}[1] 
 {\begin{itemize} \item[(#1)~~]}{\end{itemize}}
%\makeatother
\newcommand{\EndPage}{\vfill \coursenum ~ \coursename ~
 (\examyear) \hfill Page \thepage\/ of \pagetotal \newpage}

%\newcommand{\ICLOGO}{\begin{figure}[!t]
%  \includegraphics[totalheight= 1.1cm] {imperial.eps} 
%  \end{figure} \par }
%%%%
% This requires imperial.eps in the graphics input path eg 
% h:/images/imperial.eps 
%%%% 


\newcommand{\draft}{\begin{flushright} \begin{tabular}{ll}
  Course:   & \coursenum\\ Setter:   & \setter \\
  Checker:  & \checker \\  Editor:   & \editor \\
  External: & \external \\ Date:     & \today
  \end{tabular} \end{flushright} \vfill \par}

\newcommand{\sigs}{\vfill \par
  \fbox{\begin{minipage}{0.98\textwidth}{~ \\[4mm]
  \hspace*{3mm} Setter's signature \hfill
  Checker's signature \hfill
  Editor's signature~~~~ \\[4mm]
  \hspace*{3mm} \dotfill \hfill \dotfill \hfill \dotfill ~~~~
  \\[2mm] ~} \end{minipage}}}

\newcommand{\fpageDraft}  % FOR DRAFT FRONT PAGE
{\sf % \ICLOGO
 \begin{center} IMPERIAL COLLEGE LONDON \par
 \draft
 BSc and MSci EXAMINATIONS (MATHEMATICS) \par 
 \exammonth~ \examyear \vfill \par
 \Large \coursenum \bigskip \par
 \coursename \bigskip \par
 \end{center} \vfill \par \normalsize
 \sigs  \par \EndPage}

\newcommand{\fpage}
{\sf %\ICLOGO
  \begin{center} %UNIVERSITY OF LONDON \par
  BSc and MSci EXAMINATIONS (MATHEMATICS) \par
  \exammonth~ \examyear \bigskip \par
  This paper is also taken for the relevant examination
  for the Associateship. 
  \vfill \par
  \Large \coursenum \bigskip \par
  \coursename  \normalsize \bigskip \par \vfill
  \sigs\end{center}
  \par \vfill \vfill
  \copyright ~\examyear~ Imperial College London
  \hfill \coursenum \hfill Page 1 of \pagetotal
  \newpage}

% Personally entered commands:


% use \begin{Question}{1} ... \end{Question} and
%     \begin{Part}{a} ... \end{Part} 
% use \marb{} to allocate marks and \seen \unseen etc. 
% manually put in \EndPage for correct footers.

% document text starts here
\begin{document}

\setcounter{page}{1}

 \fpage
 

 \sf

%%%%%%%%%%%%%%%%%%%%%%%%%%%%%%%%%%%%%%%%%%%%%%%%%%%%%%%%%%%
\begin{Question}{1}
  \begin{Part}{a}\seen
    Label the triangle vertices $v_1$, $v_2$, $v_3$, with edges
    $\Pi_1$ joining $v_1$ and $v_2$, $\Pi_2$ joining $v_2$ and $v_3$,
    and $\Pi_3$ joining $v_3$ and $v_1$. Let $L_i$ be the
    non-degenerate affine function that vanishes on $\Pi_i$, for
    $i=1,2,3$. Now let $v$ be the quadratic polynomial that vanishes
    on all of the vertices and edge centres. If we can show that
    $v\equiv 0$, then we can conclude that $\mathcal{N}$ determines
    $P$. Consider $v$ restricted to $\Pi_1$. Since $v$ vanishes at 3
    points on $\Pi_1$ (two vertices and an edge centre), then by
    the fundamental theorem of algebra, $v$ vanishes on the entire of $\Pi_1$.
    Hence, $v(x)=L_1(x)Q_1(x)$ where $Q_1$ is a polynomial of degree at
    most 1. Similarly, $v$ vanishes everywhere on $\Pi_2$, so $Q_1$ must
    vanish everywhere on $\Pi_2$, except possibly where $\Pi_2$ intersects
    $\Pi_1$, since $L_1(x)$ is zero there. However, $Q_1$ is continuous
    so must vanish everywhere on $\Pi_2$. Hence, $Q_1=L_2c$, where $c$ is a
    constant, and so $v=L_1L_2c$. Neither $L_1$ nor $L_2$ vanish
    on the edge centre of $\Pi_3$, but $v$ vanishes there, so $c$ must be
    zero, and so $v\equiv 0$ as required.
    \amarks{7}
  \end{Part}
  \begin{Part}{b}\simseen
    To each vertex we associate the nodal variable corresponding to
    point evaluation at that vertex. To each edge we associate the nodal
    variable corresponding to point evaluation at the centre of that edge.

    To check it is $C^0$, we need to check that we can (a) recover the
    value of the function at each vertex using nodal variables from
    that vertex, and (b) recover the value of the function at each
    edge using nodal variables from the closure of that edge.  Showing
    (a) is immediate. For (b), on the closure of each edge we have two
    vertex values and an edge value. That is enough to recover the
    value of the quadratic function restricted to the edge.
    \amarks{7}
  \end{Part}
  \begin{Part}{c}\unseen
    To give a counter example, consider the unit square subdivided into
    two triangles by the diagonal line $x=y$. We consider the finite
    element function equal to $x-y$ in the top-left triangle, and
    equal to zero in the bottom right. This is a piecewise polynomial
    of degree at most 2 (even though it is only degree 1). The derivative
    is equal to $(1,-1)$ in the top-left and $(0,0)$ in the bottom right,
    and hence it is discontinuous. Hence, the function is only in $C^0$
    and not in $C^1$, as required.
    \amarks{6}
  \end{Part}
\end{Question}
\EndPage

\begin{Question}{2}
  \begin{Part}{a}\simseen
    Multiplication by test function, and integration by parts gives
    \begin{equation}
      \int_{\Omega} \epsilon uv + \nabla u \cdot \nabla v \diff x
      - \int_{\partial\Omega} v\pp{u}{n}\diff S
      = \int_{\Omega}v\exp(xy)\diff x.
    \end{equation}
    Then, application of the boundary condition leads to the
    variational problem: find $u\in V$ such that
    \begin{equation}
      a(u,v) := \int_{\Omega} \epsilon uv + \nabla u \cdot \nabla v \diff x
      = F(v) := \int_{\Omega}v\exp(xy)\diff x, \quad \forall v \in V,
    \end{equation}
    where $V\subset H^1$ is a continuous finite element space.
    \amarks{6}
  \end{Part}
  \begin{Part}{b}\simseen
    The Schwarz inequality gives
    \begin{align}
      a(u,v) \leq & \epsilon \|u\|_{L^2}\|v\|_{L^2} + |u|_{H^1}|v|_{H^1}, \\
      \leq & \|u\|_{L^2}\|v\|_{L^2} + |u|_{H^1}|v|_{H^1}, \\
      \leq & 2\|u\|_{H^1}\|v\|_{H^1},
    \end{align}
    so the continuity constant is bounded above by 2 (it is actually
    1 from Young's inequality), but we don't require sharper estimates
    here.

    For coercivity,
    \begin{equation}
      u(v,v) = \epsilon \|v\|_{L^2}^2 + |v|_{H^1}^2
      \geq \epsilon\|v\|_{H^1}^2,
    \end{equation}
    so the coercivity constant is $\epsilon$.
    \amarks{6}
  \end{Part}
  \begin{Part}{c}\unseen
    Combining Ce\'a's lemma and the interpolation error
    estimate, we get
    \begin{equation}
      \|u-u_h\|_{H^1}\leq C\frac{2}{\epsilon}h\|u\|_{H^1}.
    \end{equation}
    As $\epsilon\to 0$, the error estimate becomes unbounded.
    \bmarks{8}
  \end{Part}
\end{Question}
\EndPage

%%%%%%%%%%%%%%%%%%%%%%%
\begin{Question}{3}
  \begin{Part}{a}\unseen
    We work with the geometric decomposition, assigning
    the two tangential components on each edge to that edge.
    Then, we require that neighbouring cells have the same
    nodal variable values from that edge (taking care
    of orientation). Since the tangential component of the
    function is a linear scalar function restricted to the edge,
    and the nodal variables give two values of that function on the
    edge, the tangential component is determined from those values
    and will be continuous across the edge.
    \cmarks{6}
  \end{Part}
  \begin{Part}{b}\unseen
    If we have $\phi\in P_2$, then $\phi$ is continuous across an edge.
    Therefore the tangential component of the derivative is the same
    on either side of the edge. Further, the derivative is a linear
    vector field, so $\nabla\phi\in V$.
    \dmarks{8}
  \end{Part}
  \begin{Part}{c}\simseen
    For $u\in V$, the weak curl is $Du\in L^2$ such that
    \begin{equation}
      Du|_K = \nabla^{\perp}\cdot u|_K,
    \end{equation}
    for each triangle $K$ in the mesh. To check that this is indeed
    the weak curl, take a $C^\infty_0$ function (infinitely
    differentiable and all derivatives vanish on the boundary) $\phi$,
    and compute
    \begin{align}
      -\int_\Omega \phi Du \diff x & = -\sum_K \int_K
      \phi Du \diff x, \\
      & = \sum_K \int_K  \nabla^\perp\phi\cdot u \diff x
      - \sum_K \int_{\partial K} \phi u\cdot n_K^\perp \diff S, \\
     & = \int_\Omega  \nabla^\perp\phi\cdot u \diff x
      - \sum_K \int_{\partial K} \phi (u^+\cdot (n^+)^{\perp} + u^-\cdot (n^-)^{\perp}) \diff S, \\
      & = \int_\Omega  \nabla^\perp\phi\cdot u\diff x
      - \int_{\partial \Omega} \underbrace{\phi=0}
      (u^+\cdot (n^+)^{\perp} + u^-\cdot (n^-)^{\perp}) \diff S, \\
      - \int_{\Gamma} \phi \underbrace{(u^+\cdot (n^+)^{\perp} + u^-\cdot (n^-)^{\perp})}_{=0} \diff S, \\
      & = \int_\Omega  \nabla^\perp\phi\cdot u \diff x,
    \end{align}
    where $n_K$ is the outward pointing normal to each triangle $K$,
    and the facet integrals vanish due to continuity of the tangential
    component. This matches the definition of the weak curl.
    \cmarks{6}
  \end{Part}
\end{Question}

\EndPage

%%%%%%%%%%%%%%%%%%%%%%%%%%%%%%%
\begin{Question}{4}
  \begin{Part}{a}\unseen
    We have
    \begin{align}
      \nonumber
    \int_\Omega u^{k+1}v + \nabla u^{k+1} \cdot \nabla v \diff x = & 
    \int_\Omega u^{k}v + \nabla u^{k} \cdot \nabla v \diff x \\
& \qquad    + \mu\left(
    G[v] -
    \int_\Omega a(x) u^kv + b(x)\nabla u^k \cdot \nabla v \diff x\right)
    , \quad
    \forall v \in V,
    \end{align}
    Assuming that the limit exists, we take the limit,
    and get
        \begin{align}
      \nonumber
    \int_\Omega u^*v + \nabla u^* \cdot \nabla v \diff x = & 
    \int_\Omega u^*v + \nabla u^* \cdot \nabla v \diff x \\
& \qquad    + \mu\left(
    G[v] -
    \int_\Omega a(x) u^*v + b(x)\nabla u^* \cdot \nabla v \diff x\right)
    , \quad
    \forall v \in V,
        \end{align}
        which reduces to
        \begin{equation}
          \mu\left(
          G[v] -
          \int_\Omega a(x) u^*v + b(x)\nabla u^* \cdot \nabla v \diff x\right)
          =0, \quad
          \forall v \in V,
        \end{equation}
        which implies $u^*$ solves our equation for $\mu>0$.
  \bmarks{6}
  \end{Part}
  \begin{Part}{b}\unseen
  First we use that
          \begin{equation}
          G[v] =
          \int_\Omega a(x) uv + b(x)\nabla u \cdot \nabla v \diff x
          , \quad
          \forall v \in V,
          \end{equation}
          so
          \begin{align}
            \begin{split}
            \mu\left(
            G[v] -
            \int_\Omega a(x) u^{k}v + b(x)\nabla u^k\cdot \nabla v \diff x\right) \\
            =
            \mu\left(
            \int_\Omega a(x) uv + b(x)\nabla u\cdot \nabla v \diff x      
            -\int_\Omega a(x) u^{k}v + b(x)\nabla u^k\cdot \nabla v \diff x\right) \\
            =
            \mu\left(
            \int_\Omega a(x) (u-u^k)v + b(x)\nabla (u-u^k)\cdot \nabla v \diff x\right)
            , \quad
            \forall v \in V.
            \end{split}
          \end{align}
          Then we get
          \begin{equation}
    \int_\Omega \epsilon^{k+1}v + \nabla \epsilon^{k+1} \cdot \nabla v \diff x = 
\mu
\int_\Omega a(x) \epsilon^kv + b(x)\nabla \epsilon^k \cdot \nabla v \diff x
    , \quad
\forall v \in V.
\label{eq:error}
          \end{equation}
  \bmarks{6}
\end{Part}
  \begin{Part}{c}\unseen
    Equation \eqref{eq:error} is of the form $a(u,v) = F(v)$,
    with
    \begin{equation}
      a(u,v) = \int_\Omega uv + \nabla u\cdot\nabla v \diff x,
      \quad F(v) = \mu\int_\Omega a(x)\epsilon^k v + b(x)\nabla\epsilon^k
      \cdot \nabla v \diff x.
    \end{equation}
    In this case, $a(\cdot,\cdot)$ \emph{is} the $H^1$ inner product,
    so the continuity and coercivity constants are both 1.
    We have
    \begin{equation}
      |F(v)| \leq \mu\beta \|\epsilon^k\|_{L^2}\|v\|_{L^2}
      + \mu\beta |\epsilon^k|_{H^1}|v|_{L^2}
      \leq 2\mu\beta\|\epsilon^k\|_{H^1}\|v\|_{H^1},
    \end{equation}
    (sharper estimates are possible but the question doesn't require them).
    Hence, from Lax-Milgram, we have
    \begin{equation}
      \|\epsilon^{k+1}\|_{H^1} < 2\mu\beta\|\epsilon^k\|_{H^1},
    \end{equation}
    and so the norm of the error is guaranteed to reduce if $0<\mu<1/(2\beta)$.
  \dmarks{8}
\end{Part}
\end{Question}
	


\EndPage

%%%%%%%%%%%%%%%%%%%%%%%
\begin{Question}{5}
  \begin{Part}{a}\unseen
    \begin{equation}
      (B^*p)[v] = (Bv)[p] = b(v,p).
    \end{equation}
    \mmarks{6}
  \end{Part}
  \begin{Part}{b}\seen
    We have
    \begin{equation}
      \|B^*p\|_{V'} = \sup_{0\neq v\in V}\frac{(B^*p)[v]}{\|v\|_V}
      = \sup_{0\neq v\in V}\frac{b(v,p)}{\|v\|_V} \geq \beta\|p\|_Q,
    \end{equation}
    by the inf-sup condition. Assume
    that there exists $p,q$ such that $B^*p=B^*q$. Linearity means that
    $B^*(p-q)=0$. Then,
    \begin{equation}
      0 = \|B^*(p-q)\|_{V'} \geq \beta\|p-q\|_Q \implies p=q.
    \end{equation}
    Hence $B^*$ maps each element of $Q$ to a different element of $V'$,
    i.e. it is injective.
        \mmarks{7}
  \end{Part}
  \begin{Part}{c}
    $\tilde{\nabla}$ corresponds to the operator $B^*$, followed by
    the Riesz map back into $V$. Since $B^*$ is injective, and the Riesz
    map is invertible, therefore $\tilde{\nabla}$ is also injective.
    \mmarks{7}
  \end{Part}
\end{Question}

\EndPage
{\bf Review of mark distribution:}\\ 
\totamarks\\
 \totbmarks\\
 \totcmarks\\
 \totdmarks\\
  \totmarks
  \\  \totMmarks %Include Mastery marks
 \label{pagetot}
\EndPage


%%%%%%%%%%%%%%%%%%%%%%%%%%%%%%%%%%
\end{document}




