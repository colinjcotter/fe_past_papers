
\documentclass[12pt]{article}

\usepackage{graphicx,amssymb,amsmath,nicefrac,ifthen,calc}
\usepackage[T1]{fontenc}
\usepackage{lmodern}

%%%%%%%%%%%%%%%%%%%%%%%%%%%%%%%%%%%
%  Fill in the following details  %
%%%%%%%%%%%%%%%%%%%%%%%%%%%%%%%%%%%

\newcommand{\coursenum}{MATH97095} %{e.g: MATH50001, or: MATH96057/MATH97006/MATH97171 for multiple year courses}
\newcommand{\coursename}{Finite Elements} %{e.g: Mathematical Logic} 
\newcommand{\setter}{Cotter}     % Insert surname(s) of exam setter(s) 
\newcommand{\checker}{Ham} % Insert surname of exam checker 
\newcommand{\editor}{Wu} % Insert surname of exam editor
\newcommand{\external}{external} % Insert surname of external examiner
\newboolean{tables}
\setboolean{tables}{false} %set to true if statistical tables are provided and attach require tables

% Choose an option for time allowed and number of questions %

\newcommand{\examlength}%Uncomment one of the following 4 options, or adjust.
%{2~Hours}
%{2.5~Hours}
{2 Hours for MATH96 paper; 2.5 Hours for MATH97 papers}
%{3~Hours} %For 2-term modules

\newcommand{\numQ}%Uncomment one of the following 4 options, or adjust.
%{4~Questions}
%{5~Questions}
{4~Questions (MATH96 version); 5~Questions (MATH97~versions)}
%{6~Questions} %For 2-term modules

\newcommand{\version}% Uncomment one of the following, or adjust
{Final Version}
%{Version for External Examiner}
%{Final version}


\newcommand{\suitability}% Uncomment one of the options below
{Is the paper suitable for resitting students from previous years:
%Yes
No, because the Mastery material is on a different topic in Question 5.
}



%%%%%%%%%%%%%%%%%%%%%%%
%The following details about the difficulty level of the questions must be completed; also include this information on your mark-scheme.
%%%%%%%%%%%%%%%%%%%%%%%

\newcommand{\CatA}% Indicate, with marks, the most straightforward 40 percent of the marks available in the non-mastery questions
{1a:8, 1bi:6, 3a:6, 4a:6, 4b:8
}
\newcommand{\CatB}% Indicate the next most challenging 25 percent of the marks in the non-mastery questions
{1bii:6, 2a:6, 2b:7
}
\newcommand{\CatC}%Indicate the next most challenging 15 percent of the marks in the non-mastery questions
{2c:7, 4c:6
}
\newcommand{\CatD}% Indicate the most challenging 20 percent of marks on the paper in the non-mastery questions
{3b:6, 3c:8
}



%%%%%%%%%%%%%%%%%%%%%%%%%%%%%%%%%%%%%
% Insert your macros %
%%%%%%%%%%%%%%%%%%%%%%%%%%%%%%%%%%%%%

\newcommand{\pp}[2]{\frac{\partial #1}{\partial #2}} 
\newcommand{\dede}[2]{\frac{\delta #1}{\delta #2}}
\newcommand{\dd}[2]{\frac{\diff#1}{\diff#2}}
\newcommand{\dt}[1]{\diff\!#1}
\def\MM#1{\boldsymbol{#1}}
\DeclareMathOperator{\diff}{d}

% Go to `Exam text starts from here' below to type in your questions. %

%%%%%%%%%%%%%%%%%%%%%%%%%%%%%%%%%%%%%%%%%%%%%%%%%%%%%%%%%%%%%%%%%
%%%%%%%%%%%%%%%%%%%%%%%%%%%%%%%%%%%%
% Ignore the following details     %
%%%%%%%%%%%%%%%%%%%%%%%%%%%%%%%%%%%%

\newcommand{\examyear}{2021}
\newcommand{\exammonth}{May -- June} 
\newcommand{\examdate}{Wednesday, 5th May 2021} %{Tuesday, 26th May 2021} 
\newcommand{\examtime}{09:30 -- 11:30} 
\newboolean{onebook}
\setboolean{onebook}{true} %set to true if all answers should be in one answer book

%%%%%%%%%%%%%%%%%%%%%%%%%%%%%%%%%%%%
% setup a counter to count up marks in a question
\newcounter{tmarks}
\setcounter{tmarks}{0}

% Define the command for formatting the marks available for the question


\newcommand{\qmarks}[1]{\addtocounter{tmarks}{#1}\ifthenelse{\equal{#1}{1}}{\parbox{0.1in}{\ }\hfill{(#1 mark)}}{\parbox{0.1in}{\ }\hfill{(#1 marks)}}}

\newcommand{\qqmarks}[1]{\addtocounter{tmarks}{#1}\ifthenelse{\equal{#1}{1}}{\newline\parbox{0.1in}{\ }\hfill{(#1 mark)}}{\newline\parbox{0.1in}{\ }\hfill{(#1 marks)}}}

% define a total marks command for the end of the question
\newcommand{\totmarks}{\ifthenelse{\equal{\value{tmarks}}{1}}{\parbox{0.1in}{\ }\hfill{(Total: \arabic{tmarks} mark)}}{\parbox{0.1in}{\ }\hfill{(Total: \arabic{tmarks} marks)}}}





\pretolerance=100000
\setlength{\topmargin}{-15mm}
\setlength{\textheight}{245mm}
\setlength{\textwidth}{178mm}
\setlength{\oddsidemargin}{-10mm}
\setlength{\evensidemargin}{-10mm}
\setlength{\marginparwidth}{1cm}
\setlength{\parskip}{1.1ex}
\setlength{\parindent}{0ex}
\renewcommand{\baselinestretch}{1.1}

\pagestyle{empty}

\newenvironment{Question}[1] 
 {\begin{itemize} \item[\large #1.~~]}{\end{itemize}\totmarks\setcounter{tmarks}{0} \medskip}
 
\newcommand{\EndPage}{
	\vfill \coursenum ~ \coursename ~
	(\examyear) \hfill Page \thepage \newpage
	}

%\newcommand{\ICLOGO}{
%	\begin{minipage}{0.7\textwidth}
%  \includegraphics[height= 1.4cm]{imperial.pdf} 
%  \end{minipage}\hfill \parbox[r][1.4cm][t]{0.2\textwidth}{\hfill \coursenum} \par 
%	}
	
\newcommand{\BeginParts}{\begin{itemize}} 
\newcommand{\Part}[1]{\item [(#1)~~]} 
\newcommand{\EndParts}{\end{itemize}} 

%%%%
% This requires imperial.pdf in the graphics input path eg 
% h:/images/imperial.pdf 
%%%% 

\newcommand{\draft}{
	\begin{flushleft} 
	\begin{tabular}{ll}
  Module:   & \coursenum\\ Setter:   & \setter \\
  Checker:  & \checker \\  Editor:   & \editor \\
  External: & \external \\ Date:     & \today\\
  Version: & \version
  \end{tabular} 
  \end{flushleft} 
  \vfill \par
	}


\newcommand{\fpagedraft}{  % FOR DRAFT FRONT PAGE with SIGS
%	\ICLOGO
	\begin{center} 
	\draft
	BSc, MSci and MSc EXAMINATIONS (MATHEMATICS) \par 
	\exammonth~ \examyear 
	
	\medskip
	
	\large \coursenum \quad
	\coursename 
	\end{center}
	\medskip
	
	\textit{The following information must be completed:}
	
	\medskip
	
	\textbf{\suitability}
	
	\medskip
	
	\textbf{Category A marks: available for basic, routine material (excluding any mastery question)\\(40 percent = 32/80 for 4 questions):}\newline
	\CatA
	
	\medskip
	
	\textbf{Category B marks: Further 25 percent of marks (20/ 80 for  4 questions) for demonstration of a sound knowledge of a 
	good part of the material and the solution of straightforward problems and examples with reasonable accuracy (excluding mastery question):} \newline
	\CatB
	
	\medskip
	
	\textbf{Category C marks: the next 15 percent of the marks (= 12/80 for 4 questions)  for parts of questions at the high 2:1  or 1st class level (excluding mastery question):}\newline
	\CatC
	
	\medskip
	
	\textbf{Category D marks: Most challenging 20 percent (16/80 marks for 4 questions) of the paper (excluding mastery question):}\newline
	\CatD	
	
	
	
	 \vfill \par \normalsize
	 
	
	
	\textit{Signatures are required for the final version:}
		\sigs  \par \vfill 
	\copyright ~\examyear~ Imperial College London
	\hfill \coursenum \hfill Temporary cover page
	\newpage
	}


	
	
\newcommand{\sigs}{ % Signatures
	\vfill \par 
  \fbox{\begin{minipage}{0.98\textwidth}{~ \\[4mm]
  \hspace*{3mm} Setter's signature \hfill
  Checker's signature \hfill
  Editor's signature~~~~ \\[4mm]
  \hspace*{3mm} \dotfill \hfill \dotfill \hfill \dotfill ~~~~
  \\[2mm] ~} \end{minipage}}
	}
	


	
\newcommand{\fpage}{ % FRONT PAGE 
%	\ICLOGO
%  
\newlength{\myl}
\settowidth{\myl}{\sc Temporary front page}
\newlength{\myll}
\setlength{\myll}{\textwidth - \myl}
\begin{center}
 \large{\sc \rule[4pt]{0.45\myll}{0.5pt}~Temporary front page~\rule[4pt]{0.45\myll}{0.5pt}}
 \end{center}
 %
 \bigskip
  \begin{center}
  BSc, MSc and MSci EXAMINATIONS (MATHEMATICS) \par
  \exammonth~ \examyear \bigskip \par
  This paper is also taken for the relevant examination
  for the Associateship of the \\ Royal College of Science. 
	\end{center}
	\begin{center}
	\coursename
	\end{center}
	\renewcommand{\arraystretch}{1.3}
	\vfill \par
	\fbox{\parbox{\textwidth}{
	\begin{tabular}{l}
	Date: \examdate \\ 
	Time: \examtime \\
	Time Allowed: \examlength \\
	This paper has {\em \numQ}. \\
	%\ifthenelse{\boolean{onebook}}
	%{Candidates should use ONE main answer book.}
	%{Candidates should start their solutions to each question in a new main answer book.}\\
	%Candidates should use ONE main answer book.\\
	%Supplementary books may only be used after the relevant main book(s) are full. \\
	\ifthenelse{\boolean{tables}}{Statistical tables are provided.}{Statistical tables will not be provided.}\\
	\end{tabular}}}
	\renewcommand{\arraystretch}{1.5}
	\begin{itemize}
	%\item DO NOT OPEN THIS PAPER UNTIL THE INVIGILATOR TELLS YOU TO.
	%\item Affix one of the labels provided to each answer book that you use, 
	%but DO NOT USE THE LABEL WITH YOUR NAME ON IT.
	\item	Credit will be given for all questions attempted.
	\item Each question carries equal weight.
	%\item Calculators may not be used.
	\end{itemize}
	\vfill 
	\copyright ~\examyear~ Imperial College London
	\hfill \coursenum \hfill Page 1 of \pageref{ptotal}
	\newpage
	}
%%%%%%%%%%%%%%%%%%%%%%%%%%%%%%%%%%%%%%%%%%%%%%%%%%%%%%%%%%%%%%%%
%%%%%%%%%%%%%%%%%%%%%%%%%%%%%%%%%%%%%%%%%%%%%%%%%%%%%%%%%%%%%%%%
% Exam text starts from here
%%%%%%%%%%%%%%%%%%%%%%%%%%%%%%%%%%%%%%%%%%%%%%%%%%%%%%%%%%%%%%%%%
%%%%%%%%%%%%%%%%%%%%%%%%%%%%%%%%%%%%%%%%%%%%%%%%%%%%%%%%%%%%%%%%%

% Use \begin{Question}{1} ... \end{Question} and
%     \BeginParts \Part{a} \EndParts
%
% Avoid splitting a question across two pages
%
% Insert number of marks for each part or sub-part using \qmarks{}. Use \qqmarks{} if this results in a bad line break.
%
% Manually put in \EndPage at the end of every page for the correct footers.

\begin{document}
\sffamily
\fpagedraft % this produces the signature page 
\setcounter{page}{1}
\fpage % this produces the front page with rubric


\begin{Question}{1}
\BeginParts
\Part{a}
Consider the finite element $(K,P,\mathcal{N})$ given by
\begin{enumerate}
\item $K$ is the $1\times 1$ square, with bottom-left corner at $(0,0)$.
\item $P$ is the polynomial space spanned by $\{1,x,y,xy\}$.
\item $\mathcal{N}=(N_1,N_2,N_3,N_4)$ where $N_i(p)=p(z_i)$ and
  $(z_1,z_2,z_3,z_4)$ are the four corners of the square.
\end{enumerate}
Find the nodal basis for this finite element. (You may use a
computational linear algebra package such as Numpy or Matlab to invert
matrices but you must write the matrices that you are computing with
in your solution.)
\qmarks{8}
\Part{b}
Consider a finite element $(K,P,\mathcal{N})$ with 
\begin{enumerate}
\item $K$ is the triangle with vertices at $z_1=(0,0)$, $z_2=(1,0)$,
  and $z_3=(0,1)$.
\item $P$ is the polynomial space spanned by $\{1,x,y,xy(1-x-y)\}$,
\end{enumerate}
and nodal basis
\begin{align} \nonumber
  \psi_1 = x-9xy(1-x-y), \, \psi_2=y-9xy(1-x-y),\\ \psi_3=1-x-y-9xy(1-x-y), \, \psi_4=27xy(1-x-y).
\end{align}
\BeginParts
\Part{i}
Find a set of nodal variables $\mathcal{N}$ that corresponds
to this nodal basis, justifying your answer.
\qmarks{6}
\Part{ii}
Provide a $C^0$ geometric decomposition for this element, explaining
why it has the specified continuity.
\qmarks{6} \EndParts \EndParts
\end{Question}

\EndPage
\begin{Question}{2}
  In this question we consider the Poisson equation
  \begin{equation}
    -\nabla^2 u = f,
  \end{equation}
  on a convex domain $\Omega$ with $u=0$ on $\partial\Omega$. We
  assume that $f$ is such that $u \in H^2(\Omega)$ but $u \notin
  H^3(\Omega)$.
  \BeginParts
  \Part{a} Consider Theorem 5.30 of the notes. Explain why this theorem
  is not sufficient for estimating the convergence rate for finite
  element discretisation for this problem when $k=2$.
  \qqmarks{6}
  \Part{b} For $i=1$, propose and prove a modification of Lemma 5.28 for this case.
  \qqmarks{7}
  \Part{c} For $i=1$, propose and justify (referring to existing proofs in the notes) a modification of Theorem 5.30 for this case. Comment on the difference in the estimate
  compared to the case $u\in H^3(\Omega)$.  \qmarks{7} \EndParts
\end{Question}

\EndPage

\begin{Question}{3} 
\BeginParts
\Part{a}
Write a finite element variational problem for the following equation,
\begin{equation}
  -\nabla^2 u = 0, \quad \frac{\partial u}{\partial n}=g \mbox{ on } \partial\Omega,
\end{equation}
describing the types of finite element spaces that should be used to
ensure a unique solution.
\qqmarks{6}
\Part{b}
Consider the finite element discretisation in the case where $\Omega$
is a $1\times 1$ square and $g=\exp(\cos(x)\cos(y+x))$. Explain why
your variational problem is not possible to implement exactly on a
computer when using numerical quadrature, and propose a modification
that is.  \qqmarks{6}
\Part{c}
Adjust the statement and proof of C\'ea's Lemma to accommodate your
modification from (b), and comment on the conditions for convergence of the
finite element solution to the exact solution as the mesh is refined.
(Hints: start from the triangle inequality for
$\|u-v+v-u_h\|_{H^1(\Omega)}$ for $v\in V_h$ and then work with
the term $\|v-u_h\|_{H^1(\Omega)}$.)
\qqmarks{8}
\EndParts
\end{Question}

 \EndPage
\begin{Question}{4}
  In this question we consider the domain displayed in Figure
  \ref{fig:boxbox}. The domain $\Omega$ is the entire square area
  shaded grey, with outer boundary $\partial\Omega$ shown as a
  continuous black line. Inside the domain is a smaller square,
  denoted $\Omega_0$, with boundary $\Gamma$, shown as a dashed black
  line. We define $\Omega_1$ to be the complement of $\Omega_0$ in
  $\Omega$.

  \begin{figure}
    \centerline{\includegraphics[width=6cm]{boxbox}}
    \caption{\label{fig:boxbox} Domain for Question 4.}
  \end{figure}
  
  We consider the following problem: find $u$ such
  that
  \begin{align}
    -\nabla^2 u = 0, \mbox{ in $\Omega_0$ and $\Omega_1$}, \\
    u = 0, \mbox{ on $\partial\Omega$}, \\
    \left.\frac{\partial u}{\partial n}\right|_{\partial\Omega_0}
    + \left.\frac{\partial u}{\partial n}\right|_{\partial\Omega_1\cap \Gamma} = 2,
  \end{align}
  where for a domain $\Omega_i$, $\frac{\partial u}{\partial
    n}|_{\Omega_i}$ is the value of the normal component of the
  derivative restricted to $\Omega_i$, using the outward pointing
  normal to $\partial\Omega_i$. Note that since the outward pointing
  normals to $\partial\Omega_0$ and $\partial\Omega_1\cap \Gamma$ are
  equal and opposite, this condition on $\Gamma$ indicates a
  discontinuity in the normal derivative of $u$.
  \BeginParts
\Part{a}
Using the continuous Lagrange finite element space of degree $k$,
formulate a finite element discretisation for this problem.
\qmarks{6}
\Part{b}
Show that the finite element discretisation has a unique solution, and
provide a constant $\gamma$ such that the finite element solution satisfies
\begin{equation}
  \|u_h\|_{H^1(\Omega)} \leq \gamma,
\end{equation}
independently of $h$. You may quote results from the course notes
without proof.
\qmarks{8}
\Part{c}
Let $u_h$ be the numerical solution on a mesh consisting of squares
subdivided into right angled triangles, with square edge length $h$,
and let $u$ be the exact solution of the problem. Discuss the
applicability of the bound on $\|u_h-u\|_{H^1(\Omega)}$ that we
studied in the course.
\qmarks{6} \EndParts
\end{Question}


%% If you have a Mastery Question, then this should be included as an additional question 
%% on a new page. Otherwise, comment out the following three lines:
\EndPage
\begin{Question}{5}
  \begin{figure}
  \centerline{\includegraphics[width=6cm]{p1p1}}
  \caption{\label{fig:p1p1} Example function values for Question 5.}
\end{figure}
 \BeginParts
\Part{a}
Let $V=(H^1(\Omega))^2$ and $Q=\mathring{L}^2(\Omega)$, where $\Omega$
is a convex polygonal domain. Let $b:V\times Q\to \mathbb{R}$ be the
bilinear form
\begin{equation}
  b(v,q) = \int_\Omega q\nabla\cdot v \diff x.
\end{equation}
Assuming the
result that the divergence is surjective from $V$ to $Q$, show
that $b$ satisfies the inf-sup condition
\begin{equation}
  \inf_{q\in Q}\sup_{v\in V}\frac{b(v,q)}{\|v\|_V\|q\|_Q} \geq \beta,
\end{equation}
for some constant $\beta$.
\qqmarks{5}
\Part{b}
Define the operator $\delta:Q \to V$ by
\begin{equation}
  \int_{\Omega} w\cdot \delta q \diff x = b(w, q), \quad \forall w \in V.
\end{equation}
Show that the kernel of $\delta$, $\mathrm{Ker}(\delta)$, is empty.
(Hint: find a suitable choice of test function.)
\qqmarks{5}
\Part{c}
Let $V_h\subset V$ and $Q_h\subset Q$ be finite element spaces chosen for
the discretisation of Stokes' equation.  Let $\Pi_h:V\to V_h$ satisfy
condition 1 of Fortin's trick, i.e.
\begin{align}
  b(v-\Pi_hv, q) = 0, \quad \forall v\in V, \, q\in Q_h.
\end{align}
Define the discrete operator $\delta_h:Q_h \to V_h$ by
\begin{equation}
  \int_{\Omega} w\cdot \delta_h q \diff x = b(w, q), \quad \forall w \in V_h.
\end{equation}
Show that $\mathrm{Ker}(\delta_h)\subseteq \mathrm{Ker}(\delta)$.\\
\qmarks{5}\\
       (Question continues on the following page.)
\Part{d}
Consider a mesh consisting of squares subdividing into right angle
triangles by joining the top left vertex and the bottom right vertex
of each square, and consider the discretisation for Stokes with
continuous linear Lagrange elements for each component of the velocity
and continuous linear Lagrange elements for the pressure.

By considering the function $p\in Q_h$ taking vertex values in an
alternating pattern as indicated in Figure \ref{fig:p1p1}, show that
$\mathrm{Ker}(\delta_h)\not\subseteq \mathrm{Ker}(\delta)$ in this
case. \\
Do $V_h$ and $Q_j$ satisfy condition 1 of Fortin's trick?
\qmarks{5}
\EndParts
 \end{Question}


\label{ptotal}
\EndPage
\end{document}
