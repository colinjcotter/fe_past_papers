
\documentclass[12pt]{article}

\usepackage{graphicx,amssymb,amsmath,nicefrac,ifthen,calc}
\usepackage[T1]{fontenc}
\usepackage{lmodern}

%%%%%%%%%%%%%%%%%%%%%%%%%%%%%%%%%%%
%  Fill in the following details  %
%%%%%%%%%%%%%%%%%%%%%%%%%%%%%%%%%%%

\newcommand{\coursenum}{M3A47, M4A47, M5A47} %{e.g: M1S, or: M3S14/M4S14/M5S15 for multiple year courses}
\newcommand{\coursename}{Finite elements: analysis and implementation} %{e.g: Probability and Statistics} 
\newcommand{\setter}{Cotter}     % Insert surname of exam setter 
\newcommand{\checker}{Ham} % Insert surname of exam checker 
\newcommand{\editor}{Walton} % Insert surname of exam editor
\newcommand{\external}{external} % Insert surname of external examiner
\newboolean{tables}
\setboolean{tables}{false} %set to true if statistical tables are provided and attach require tables

% Choose an option for time allowed and number of questions %

\newcommand{\examlength}%Uncomment one of the following 3 options, or adjust.
%{2~Hours}
%{2.5~Hours}
{2 Hours for M3 paper; 2.5 Hours for M4/5 paper}

\newcommand{\numQ}%Uncomment one of the following 3 options, or adjust.
%{4~Questions}
%{5~Questions}
{4~Questions (M3 version); 5~Questions (M4/5~version)}

\newcommand{\version}% Uncomment one of the following, or adjust
{Draft version for checking}
%{Version for External Examiner}
%{Final version}


\newcommand{\suitability}% Uncomment one of the options below
{Is the paper suitable for resitting students from previous years:
  Yes (they will need notification that the format has changed from
  4 to 5 questions as we previously had no mastery question as only
  offered to 4th years/MSc/MRes)
%No (provide details)
}



%%%%%%%%%%%%%%%%%%%%%%%
%The following details about the difficulty level of the questions must be completed; also include this information on your mark-scheme.
%%%%%%%%%%%%%%%%%%%%%%%

\newcommand{\CatA}% Indicate, with marks, the most straightforward 40 percent of the marks available in the non-mastery questions
{1(a) 10 marks; 2(a) 10 marks; 2(b) 10 marks. (total 30)}
\newcommand{\CatB}% Indicate the next most challenging 25 percent of the marks in the non-mastery questions
{1(b) 3 marks; 1(c,i)  2 marks;  3(a) 10 marks;  4(b) 5 marks. (total 20)}
\newcommand{\CatC}%Indicate the next most challenging 15 percent of the marks in the non-mastery questions
{3(b) 10 marks; 4(a) 5 marks (total 15)}
\newcommand{\CatD}% Indicate the most challenging 20 percent of marks on the paper in the non-mastery questions
{1(c,ii) 5 marks; 4(c) 5 marks; 4(d) 5 marks. (total 15)
}



%%%%%%%%%%%%%%%%%%%%%%%%%%%%%%%%%%%%%
% Insert your macros %
%%%%%%%%%%%%%%%%%%%%%%%%%%%%%%%%%%%%%

\newcommand{\seen}{{\bfseries SEEN\\}}
\newcommand{\similar}{{\bfseries SEEN SIMILAR\\}}
\newcommand{\unseen}{{\bfseries UNSEEN\\}}
\newcommand{\exammarks}[1]{\begin{flushright}[#1 marks]\end{flushright}}%
\newcommand{\pp}[2]{\frac{\partial #1}{\partial #2}} 
\newcommand{\dede}[2]{\frac{\delta #1}{\delta #2}}
\newcommand{\dd}[2]{\frac{\diff#1}{\diff#2}}
\newcommand{\dt}[1]{\diff\!#1}
\def\MM#1{\boldsymbol{#1}}
\DeclareMathOperator{\diff}{d}


%uncomment \solnsfalse to remove solution set
\newif\ifsolns
\solnstrue
%\solnsfalse
\ifsolns
% with solutions
\usepackage{color}
\newcommand{\soln}[1]{{\bfseries Solution:} {\itshape \color{blue} #1}}
\else
% without solutions
\newcommand{\soln}[1]{}
\fi


% Go to `Exam text starts from here' below to type in your questions. %

%%%%%%%%%%%%%%%%%%%%%%%%%%%%%%%%%%%%%%%%%%%%%%%%%%%%%%%%%%%%%%%%%
%%%%%%%%%%%%%%%%%%%%%%%%%%%%%%%%%%%%
% Ignore the following details     %
%%%%%%%%%%%%%%%%%%%%%%%%%%%%%%%%%%%%

\newcommand{\examyear}{2019}
\newcommand{\exammonth}{May -- June} 
\newcommand{\examdate}{??} %{Tuesday, 26th May 2018} 
\newcommand{\examtime}{??} %{09:30 -- 11:30} 
\newboolean{onebook}
\setboolean{onebook}{false} %set to true if all answers should be in one answer book

%%%%%%%%%%%%%%%%%%%%%%%%%%%%%%%%%%%%

\pretolerance=100000
\setlength{\topmargin}{-15mm}
\setlength{\textheight}{245mm}
\setlength{\textwidth}{178mm}
\setlength{\oddsidemargin}{-10mm}
\setlength{\evensidemargin}{-10mm}
\setlength{\marginparwidth}{1cm}
\setlength{\parskip}{1.1ex}
\setlength{\parindent}{0ex}
\renewcommand{\baselinestretch}{1.1}

\pagestyle{empty}

\newenvironment{Question}[1] 
 {\begin{itemize} \item[\large #1.~~]}{\end{itemize} \medskip}
 
\newcommand{\EndPage}{
	\vfill \coursenum ~ \coursename ~
	(\examyear) \hfill Page \thepage \newpage
	}

%\newcommand{\ICLOGO}{
%	\begin{minipage}{0.7\textwidth}
%  \includegraphics[height= 1.4cm]{imperial.pdf} 
%  \end{minipage}\hfill \parbox[r][1.4cm][t]{0.2\textwidth}{\hfill \coursenum} \par 
%	}
	
\newcommand{\BeginParts}{\begin{itemize}} 
\newcommand{\Part}[1]{\item [(#1)~~]} 
\newcommand{\EndParts}{\end{itemize}} 

%%%%
% This requires imperial.pdf in the graphics input path eg 
% h:/images/imperial.pdf 
%%%% 

\newcommand{\draft}{
	\begin{flushleft} 
	\begin{tabular}{ll}
  Module:   & \coursenum\\ Setter:   & \setter \\
  Checker:  & \checker \\  Editor:   & \editor \\
  External: & \external \\ Date:     & \today\\
  Version: & \version
  \end{tabular} 
  \end{flushleft} 
  \vfill \par
	}


\newcommand{\fpagedraft}{  % FOR DRAFT FRONT PAGE with SIGS
%	\ICLOGO
	\begin{center} 
	\draft
	BSc, MSci and MSc EXAMINATIONS (MATHEMATICS) \par 
	\exammonth~ \examyear 
	
	\medskip
	
	\large \coursenum \quad
	\coursename 
	\end{center}
	\medskip
	
	\textit{The following information must be completed:}
	
	\medskip
	
	\textbf{\suitability}
	
	\medskip
	
	\textbf{Category A marks: available for basic, routine material (excluding any mastery question)\\(40 percent = 32/80 for 4 questions):}\newline
	\CatA
	
	\medskip
	
	\textbf{Category B marks: Further 25 percent of marks (20/ 80 for  4 questions) for demonstration of a sound knowledge of a 
	good part of the material and the solution of straightforward problems and examples with reasonable accuracy (excluding mastery question):} \newline
	\CatB
	
	\medskip
	
	\textbf{Category C marks: the next 15 percent of the marks (= 12/80 for 4 questions)  for parts of questions at the high 2:1  or 1st class level:}\newline
	\CatC
	
	\medskip
	
	\textbf{Category D marks: Most challenging 20 percent (16/80 marks for 4 questions) of the paper (excluding mastery question):}\newline
	\CatD	
	
	
	
	 \vfill \par \normalsize
	 
	
	
	\textit{Signatures are required for the final version:}
		\sigs  \par \vfill 
	\copyright ~\examyear~ Imperial College London
	\hfill \coursenum \hfill Temporary cover page
	\newpage
	}


	
	
\newcommand{\sigs}{ % Signatures
	\vfill \par 
  \fbox{\begin{minipage}{0.98\textwidth}{~ \\[4mm]
  \hspace*{3mm} Setter's signature \hfill
  Checker's signature \hfill
  Editor's signature~~~~ \\[4mm]
  \hspace*{3mm} \dotfill \hfill \dotfill \hfill \dotfill ~~~~
  \\[2mm] ~} \end{minipage}}
	}
	


	
\newcommand{\fpage}{ % FRONT PAGE 
%	\ICLOGO
%  
\newlength{\myl}
\settowidth{\myl}{\sc Temporary front page}
\newlength{\myll}
\setlength{\myll}{\textwidth - \myl}
\begin{center}
 \large{\sc \rule[4pt]{0.45\myll}{0.5pt}~Temporary front page~\rule[4pt]{0.45\myll}{0.5pt}}
 \end{center}
 %
 \bigskip
  \begin{center}
  BSc, MSc and MSci EXAMINATIONS (MATHEMATICS) \par
  \exammonth~ \examyear \bigskip \par
  This paper is also taken for the relevant examination
  for the Associateship of the \\ Royal College of Science. 
	\end{center}
	\begin{center}
	\coursename
	\end{center}
	\renewcommand{\arraystretch}{1.3}
	\vfill \par
	\fbox{\parbox{\textwidth}{
	\begin{tabular}{l}
	Date: \examdate \\ 
	Time: \examtime \\
	Time Allowed: \examlength \\
	This paper has {\em \numQ}. \\
	\ifthenelse{\boolean{onebook}}
	{Candidates should use ONE main answer book.}
	{Candidates should start their solutions to each question in a new main answer book.}\\
	%Candidates should use ONE main answer book.\\
	Supplementary books may only be used after the relevant main book(s) are full. \\
	\ifthenelse{\boolean{tables}}{Statistical tables are provided.}{Statistical tables will not be provided.}\\
	\end{tabular}}}
	\renewcommand{\arraystretch}{1.5}
	\begin{itemize}
	\item DO NOT OPEN THIS PAPER UNTIL THE INVIGILATOR TELLS YOU TO.
	\item Affix one of the labels provided to each answer book that you use, 
	but DO NOT USE THE LABEL WITH YOUR NAME ON IT.
	\item	Credit will be given for all questions attempted.
	\item Each question carries equal weight.
	\item Calculators may not be used.
	\end{itemize}
	\vfill 
	\copyright ~\examyear~ Imperial College London
	\hfill \coursenum \hfill Page 1 of \pageref{ptotal}
	\newpage
	}
%%%%%%%%%%%%%%%%%%%%%%%%%%%%%%%%%%%%%%%%%%%%%%%%%%%%%%%%%%%%%%%%
%%%%%%%%%%%%%%%%%%%%%%%%%%%%%%%%%%%%%%%%%%%%%%%%%%%%%%%%%%%%%%%%
% Exam text starts from here
%%%%%%%%%%%%%%%%%%%%%%%%%%%%%%%%%%%%%%%%%%%%%%%%%%%%%%%%%%%%%%%%%
%%%%%%%%%%%%%%%%%%%%%%%%%%%%%%%%%%%%%%%%%%%%%%%%%%%%%%%%%%%%%%%%%

% use \begin{Question}{1} ... \end{Question} and
%     \BeginParts \Part{a} \EndParts
%
% manually put in \EndPage at the end of every page for the correct footers.

\begin{document}
\sffamily
\fpagedraft % this produces the signature page 
\setcounter{page}{1}
\fpage % this produces the front page with rubric

\begin{Question}{1}
This question is about the equation  
\begin{equation}
  \label{eq:Poisson}
  -\nabla^2 u = f \mbox{ on }\Omega, \quad
  \pp{u}{n} = 0 \mbox{ on }\partial\Omega,
\end{equation}
where $\Omega$ is a polygonal domain with boundary $\partial\Omega$.
\BeginParts
\Part{a} Let $V$ be a continuous Lagrange finite element space
defined on a triangulation of $\Omega$.
Describe how the finite element discretisation
of \eqref{eq:Poisson} using $V$
results in a matrix-vector equation
\begin{equation}
  \label{eq:Ax=b}
  A\MM{u} = \MM{b}.
\end{equation}
\exammarks{10}
\soln{\seen
  First we develop the weak form by multiplying by a test function $v$,
  integrating by parts and removing the boundary integral due the Neumann
  boundary condition. The finite element discretisation is then:
  find $u\in V$ such that
  \[
    \int_{\Omega} \nabla v\cdot \nabla u \diff x
    - \int_{\Omega} vf \diff x = 0, \quad \forall v \in V.
  \]
  Let $\{\phi_i(x)\}_{i=1}^N$ be the nodal basis for $V$. Then expansion
  of $v$ and $u$ in the basis leads to
  \[
  \sum_{i=1}^N v_i
  \left(\sum_{j=1}^N\int_{\Omega} \phi_i(x)\phi_j(x) \diff x u_j
  - \int_{\Omega} \phi_i(x) f \diff x\right)= 0,
  \]
  but the $v$ coefficients are arbitrary, so we have \eqref{eq:Ax=b}
  with
  \[
  A_{ij} = \int_{\Omega} \nabla\phi_i\cdot\nabla \phi_j \diff x,\,
  x_i = u_i, \\
  b_i = \int_{\Omega}\phi_i f \diff x .
  \]
  }
\Part{b}
\BeginParts
\Part{i}
Show that the matrix $A$ satisfies
\begin{equation}
  \label{eq:A1=0}
  A\MM{1} = \MM{0},
\end{equation}
where $\MM{1}$ is the vector with all entries equal to 1, and
$\MM{0}$ is the zero vector.
\exammarks{2}
\soln{\unseen
  \[
  (A\MM{1})_i = \int_{\Omega} \nabla\phi_i(x)\sum_{j=1}^N
  \nabla\phi_j(x).1\diff x = \int_\Omega \nabla\phi_i(x)
  \underbrace{\nabla(1)}_{=0}\diff x = 0.
  \]
  }
\Part{ii}
Explain why this means that $A$ is not invertible.
\exammarks{1}
\soln{\unseen
  $A$ is not invertible because it has a zero eigenvalue i.e. a nullspace.
}
\EndParts
\Part{c}
\BeginParts
\Part{i}
Describe how to add an extra condition to Equation \ref{eq:Poisson},
and correspondingly to your finite element formulation, so that this
issue is removed.  \exammarks{2}
\soln{\seen We add an extra
  condition, that
  \[
  \bar{u} = \int_{\Omega} u \diff x = 0.
  \]
  Then, we replace $V$ with $\mathring{V}$ which is the subspace
  of $V$ such that $\bar{u}=0$ for all $u\in V$.
  }
\Part{ii}
Using the ``mean estimate'',
\[
  \|u-\bar{u}\|_{L^2(\Omega)} \leq C |u|_{H^1(\Omega)},
\]
where $u\in V$ and $\bar{u}$ is the mean value of $u$, explain why
Equation \eqref{eq:A1=0} cannot hold after modification.
\exammarks{5}
\soln{\unseen Let
  $A$ be the new matrix after reformulating with $\mathring{V}$
  instead of $V$, under some basis.  By contradiction: let $\MM{x}_0$
  be a non-zero vector such that $A\MM{X}_0=\MM{0}$.  Then there
  exists a corresponding non-zero $u\in \mathring{V}$ such that
  \[
  \int_{\Omega} \nabla v \cdot \nabla u\diff x = 0, \quad
  \forall v \in V.
  \]
  Taking $v=u$, we have
  \[
  0 = \int_{\Omega} |\nabla u|^2 \diff x := |u|^2_{H^1(\Omega)}.
  \]
  Since $u$ is non-zero, we have $\|u\|_{L^2} > 0$. Since
  $u \in \mathring{V}$, we have $\bar{u}=0$. Hence we have
  \[
  \|u-\bar{u}\|_{L^2(\Omega)} = \|u\|_{L^2(\Omega)} > 0.
  \]
  This contradicts the mean estimate.
  }
\EndParts
\EndParts
\end{Question}
\EndPage

\begin{Question}{2}
  \BeginParts
  \Part{a}
  Consider the finite element $(K,P,N)$, where
  \begin{itemize}
  \item $K$ is a triangle with vertices $(z_1,z_2,z_3)$.
  \item $P$ is the space of polynomials of degree 1 or less,
  \item $N=(N_1,N_2,N_3)$, where $N_i(p)=p(z_i)$, $i=1,2,3$.
  \end{itemize}
  Show that $N$ determines $P$. \exammarks{10}
        \soln{\seen We make use of the result
      that if $p(x)$ is a degree $k$ polynomial that vanishes on the
      line defined by $L(x)=0$ and $L$ is a non-degenerate affine
      polynomial, then $p(x)=L(x)q(x)$ where $q$ is a polynomial of
      degree $k-1$. \\ Let $p \in P$ such that $N_i(p)=0$,
      $i=1,2,3$. Let $L_1$ be a non-degenerate affine polynomial
      that vanishes on the line joining $z_1$ and $z_2$. Then
      the restriction of $p$ to $L_1$ vanishes at 2 points and therefore
      is zero everywhere on $L_1$ by the fundamental theorem of algebra.
      Thus $p(x)=L_1(x)q(x)$ where $q$ is a degree 0 polynomial, i.e.
      $p(x) = cL_1(x)$. We also have that $p(z_3)=0$, and $L_1(x)$
      does not vanish at $z_3$, so $c=0$ i.e. $p=0$ everywhere, hence
      $N$ determines $P$.
    }

    \Part{a}
  Consider the finite element $(K',Q,N')$, where
  \begin{itemize}
  \item $K'$ is a square with vertices $(z_1,z_2,z_3,z_4)$ (enumerated
    clockwise around the square, starting at the bottom left).
  \item $Q=\mbox{Span}\{P,xy\}$, where $P$ is the space of polynomials
    of degree 1 or less.
  \item $N'=(N_1,N_2,N_3,N_4)$, where $N_i(p)=p(z_i)$, $i=1,2,3,4$.
  \end{itemize}
  Show that $N'$ determines $Q$.
\exammarks{10}
  \soln{\similar We make use of the result
    that if $p(x)$ is a degree $k$ polynomial that vanishes on the
    line defined by $L(x)=0$ and $L$ is a non-degenerate affine
    polynomial, then $p(x)=L(x)q(x)$ where $q$ is a polynomial of
    degree $k-1$. \\ Let $p \in P$ such that $N_i(p)=0$,
    $i=1,2,3,4$. Let $L_1$ be a non-degenerate affine polynomial that
    vanishes on the line joining $z_1$ and $z_2$. Restricted to $L_1$,
    $p$ is a degree 1 polynomial, since all elements of $R$ are
    constant on $L_1$. Hence, $p(x) = L_1(x)q_1(x)$, where $q_1(x)$
    has degree 1. Similarly, let $L_2$ be the non-degenerate affine
    polynomials vanishing on the line joining $z_2$ and $z_3$. The
    restriction of $q_1$ to that line vanishes at two points and is
    therefore equal to zero everywhere on that line, and hence
    $p(x)=cL_1(x)L_2(x)$. However, $p(z_4)=0$, so $c=0$ i.e. $p:=0$
    i.e. $Q$ determines $N'$.}
  \EndParts
\end{Question}
\EndPage

\begin{Question}{3}
  Consider the interval $[a,b]$, with points
  $a=x_0,x_1,x_2,\ldots,x_{n-1}, x_n=b$.  Let $\mathcal{T}$ be a
  subdivision (i.e. a 1D mesh) of the interval $[a,b]$ into
  subintervals $I_k=[x_k,x_{k+1}]$, $k=0,\ldots,N-1$.\\
  Consider the following three elements.
  \begin{enumerate}
  \item $(K,P,N)$ where $K=I_k$, $P$ are polynomials of degree $\leq
    3$, and $N=(N_1,N_2,N_3,N_4)$ with $N_1[u]=u(x_k)$, $N_2[u]=u(x_{k+1})$,
    $N_3[u]=\int_{x_k}^{x_{k+1}} u \diff x$,
    $N_4[u]=u'((x_{k+1}+x_k)/2)$.
  \item $(K,P,N)$ where $K=I_k$, $P$ are polynomials of degree $\leq
    3$, and $N=(N_1,N_2,N_3,N_4)$ with $N_1[u]=u(x_k)$, $N_2[u]=u(x_{k+1})$,
    $N_3[u]=u'(x_k)$, $N_4[u]=u'(x_{k+1})$.
  \item $(K,P,N)$ where $K=I_k$, $P$ are polynomials of degree $\leq
    3$, and $N=(N_1,N_2,N_3,N_4)$ with $N_1[u]=u((x_{k+1}+x_k)/2)$,
    $N_2[u]=u'((x_{k+1}+x_k)/2)$, $N_3[u]=u''((x_{k+1}+x_k)/2)$,
    $N_4[u]=u'''((x_{k+1}+x_k)/2)$.
  \end{enumerate}
  \BeginParts
  \Part{a} Which of the three elements above are suitable
  for the following variational problem?\\
  Find $u\in H^1([a,b])$ such that
  \[
    \int_a^b uv + u'v'\diff x = \int_a^b fv \diff x, \quad \forall v \in
    H^1([a,b]).
  \]
  Justify your answer.
  \exammarks{10}
  \soln{\similar
    This equation requires the finite element space to be in
    $H^1([a,b])$ which requires $C^0$ finite elements. Elements 1 and
    2 can be used to make $C^0$ elements, because you can assign
    $N_1$ and $N_2$ to vertices $a$ and $b$ respectively in both cases,
    so vertex-assigned nodal variables determine the value of the function
    there. Element 3 cannot be used, as there is no $C^1$ geometric
    decomposition for it (all four
    nodal variables to determine values at $a$ and $b$ in both cases).
  }
  \Part{b}  Which of the three elements above are suitable
  for the following variational problem?\\
  Find $u\in H^2([a,b])$ such that
  \[
    \int_a^b uv + u'v' + u''v''\diff x
    = \int_a^b fv \diff x, \quad \forall v \in
    H^2([a,b]).
  \]
  Justify your answer.  \exammarks{10} \soln{\similar This equation requires the finite
    element space to be in $H^2([a,b])$ which requires $C^1$ finite
    elements. Element 2 can be used to make $C^1$ elements, because
    you can assign $N_1, N_3$ and $N_2,N_4$ to vertices $a$ and $b$
    respectively in both cases, so vertex-assigned nodal variables
    determine the value of the function and the derivative
    there.\\ Elements 1 and 3 cannot be used because the value of the
    derivatives at $a$ and $b$ require three nodal variables for each, so
    a $C^1$ geometric decomposition is not possible.
  } \EndParts
\end{Question}
\EndPage

\begin{Question}{4}
  \BeginParts
  \Part{a}  For $f \in L^2(\Omega)$, where $\Omega$ is some convex
  polygonal domain, the $L^2$ projection of $f$ into a degree $k$
  Lagrange finite element space $V$ is the function $u \in V$ such
  that
  \[
  \int_\Omega uv \diff x = \int_\Omega vf \diff x, \quad \forall
  v \in V.
  \]
  Show that $u$ exists and is unique from this definition, with
  \[
  \|u\|_{L^2} \leq \|f\|_{L^2}.
  \]
  \exammarks{5}
  \soln{\similar
    This variational problem has a bilinear form which is just the
    $L^2$ inner product. Hence it is trivially continuous and coercive
    with scaling constants equal to 1. From the Lax-Milgram theorem,
    the solution exists and is unique. Taking $v=u$, we have
    \[
    \|u\|_{L^2}^2 = \langle u, f \rangle_{L^2} \leq \|u\|_{L^2}\|f\|_{L^2},
    \]
    from Cauchy-Schwarz, and dividing both sides by $\|u\|_{L^2}$
    gives the result.
  }
  \Part{b} Show that the $L^2$ projection is mean-preserving, i.e.
  \[
  \int_{\Omega} u \diff x = \int_{\Omega} f \diff x.
  \]
  \exammarks{5}
  \soln{\unseen
    Since $V$ is a Lagrange finite element space of degree $k$, it contains
    the function $v=1$, from which we obtain the result.
  }
  \Part{c} Show that the $L^2$ projection $u$ into $V$ of $f$ is the
  minimiser over $v \in V$ of the functional
  \[
  J[v] = \int_{\Omega} (v-f)^2 \diff x.
  \]
  \exammarks{5}
  \soln{\unseen
    Method 1: solve by computing variational derivative,
    \[
    \delta J[v; \delta v] = 2\int_\Omega \delta v(v-f)\diff x = 0,
    \quad \forall \delta v \in V,
    \]
    which gives $v=u$.\\
    Method 2: by contradiction. If $u$ is not the minimiser, then
    there exists $v\in V$ with
    $J[v] \leq J[u]$.
    Then
    \begin{align*}
      J[v] = \int_\Omega (v-f)^2 \diff x &
      = \int_\Omega ((v-u) + (u-f))^2 \diff x,\\
      & = \int_\Omega (v-u)^2\diff x + \underbrace{\int_\Omega 2(v-u)(u-f)
        \diff x}_{=0\mbox{ by defn of }u} + \int_\Omega (u-f)^2\diff x,\\
      & = \|v-u\|^2_{L^2} + J[u],
    \end{align*}
    and we conclude that $\|v-u\|^2_{L^2} \leq 0$, 
    a contradiction.
  }
  \Part{d} Hence, show that
  \[
  \|u-f\|_{L^2(\Omega)} < Ch |f|_{H^1(\Omega)},
  \]
  where $h$ is the maximum triangle diameter in the triangulation used
  to construct $V$.
  \exammarks{5}
  \soln{\similar
    Since $u$ minimises the functional $J$, we have
    \begin{align*}
      \|u-f\|_{L^2(\Omega)} & = \sup_{\|v\|_{L^2(\Omega)}>0}
      \|v-f\|_{L^2(\Omega)}, \\
      & \leq \|I_hf -f\|_{L^2(\Omega)}, \\
      & \leq Ch|f|_{H^1(\Omega)},
    \end{align*}
    where $I_h$ is the nodal interpolation operator into $V$, and
    we used  the standard approximation result for $I_h$.
  }
  \EndParts
\end{Question}

%% If you have a Mastery Question, then this should be included as an additional question 
%% on a new page. Otherwise, comment out the following three lines:
 \EndPage
 
\begin{Question}{5 (Mastery)} 
  We quote the following result from lectures.  Let $K_1$ be a
  triangle with diameter $1$, containing a ball $B$. There exists a
  constant $C$ such that for $0\leq |\beta| \leq k+1$ and all $f \in
  H^{k+1}(\Omega)$,
  \begin{equation}
    \label{eq:bh}
\|D^\beta(f-Q_{k,B}f)\|_{L^2(K_1)} \leq C\|\nabla^{k+1}f\|_{L^2(K_1)},
\end{equation}
where $Q_{k,B}$ is the degree-$k$ ball-averaged Taylor polynomial of $f$.
\BeginParts
\Part{a}
Let $\mathcal{I}_{K_1}$ be the nodal interpolation operator on $K_1$
for the Lagrange finite element of degree $k$. Using the following
stability estimate
\[
  \|\mathcal{I}_Ku\|_{H^k(K_1)} \leq C\|u\|_{H^k(K_1)},
\]
when $k>1$,
together with the estimate in Equation \eqref{eq:bh},
show that when $i \leq k$, we have
\[
|\mathcal{I}_{K_1}u-u|_{H^i(K_1)} \leq C_1|u|_{H^{k+1}(K_1)}.
\]
  \exammarks{5}
\soln{\seen
  \begin{align*}
  |\mathcal{I}_{K_1}u-u|_{H^i(K_1)}^2 &\leq \|\mathcal{I}_{K_1}u-u\|_{H^{k+1}(K_1)}^2 \\
  & =
  \|\mathcal{I}_{K_1}u-Q_{k,B}u + Q_{k,B}u - u\|_{H^{k+1}(K_1)}^2 \\
  & \leq \|Q_{k,B}u-u\|_{H^{k+1}(K_1)}^2 + \|\mathcal{I}(u-Q_{k,B}u)\|_{H^{k+1}(K_1)}^2, \\
  & \leq \|Q_{k,B}u-u\|_{H^{k+1}(K_1)}^2 + C^2\|Q_{k,B}u-u\|_{H^{k+1}(K_1)}^2, \\
  & \leq (1+C^2)|u|_{H^{k+1}(K_1)}^2.
\end{align*}
}
\Part{b}
Let $K$ be a triangle with diameter $d$.
When $k>1$ and $i \leq k$, show that
\[
|\mathcal{I}_{K}u-u|_{H^i(K)} \leq d^{k+1-i}C_1|u|_{H^{k+1}(K)},
\]
where $C_1$ is a constant that depends on the shape of $K$ but
not the size.
  \exammarks{5}
\soln{\seen
Consider the change of variables $x \to \phi(x)=x/d$. This map takes
$K$ to $K_1$ with diameter 1. Then
\begin{align*}
\int_K |D^\beta(I_Ku-u)|^2 \diff x & = d^{-2|\beta|+1}\int_{K_1}|D^\beta(I_{K_1}
u\circ \phi - u\circ \phi)|^2 \diff x, \\
& \leq C_1^2d^{-2|\beta+1}\sum_{|\alpha|=k+1}\int_{K_1} |D^\alpha u\circ \phi|^2
\diff x, \\
& \leq C_1^2d^{-2|\beta+2(k+1)}\sum_{|\alpha|=k+1}\int_{K} |D^\alpha u|^2
\diff x, \\
& = C_1^2d^{2(-|\beta| + k + 1)}|u|^2_{H^{k+1}(K)},
\end{align*}
and taking the square root gives the result.
}
\Part{c} Let $\mathcal{T}$ be a triangulation such that the minimum aspect
ratio $r$ of the triangles $K_i$ satisfies $r>0$. Let $V$ be the
degree $k$ Lagrange finite element space.  Let $u\in H^{k+1}(\Omega)$.
Let $h$ be the maximum over all of the triangle diameters, assuming that with $0\leq
h<1$. Show that for $i\leq k$ and $i < 2$, the global interpolation
operator satisfies
\begin{equation}
\|\mathcal{I}_{h}u-u\|_{H^i(\Omega)} \leq Ch^{k+1-i}|u|_{H^{k+1}(\Omega)}.
\end{equation}
  \exammarks{5}
\soln{\seen
  The Lagrange finite element space is $C^0$, so the first derivatives
of $I_hu$ are defined in the finite element sense.
Then we may write (for $i< 2$)
\vspace{-3mm}
\begin{align*}
\|\mathcal{I}_{h}u-u\|_{H^i(\Omega)}^2 &=
\sum_{K\in\mathcal{T}}\|\mathcal{I}_{K}u-u\|_{H^i(K)}^2,\\
&\leq \sum_{K\in\mathcal{T}}C_Kd_K^{2(k+1-i)}|u|_{H^{k+1}(K)}^2,\\
&\leq C_{\max}h^{2(k+1-i)}\sum_{K\in\mathcal{T}}|u|_{H^{k+1}(K)}^2,\\
& = C_{\max}h^{2(k+1-i)}|u|_{H^{k+1}(\Omega)}^2,
\end{align*}
where the existence of the $C_{\max}=\max_KC_K<\infty$ is due to the
lower bound in the aspect ratio.
}
\Part{d} 
Why does this estimate not hold for $i \geq 2$?
  \exammarks{5}
\soln{\unseen
  This is because the weak second derivatives of $I_hu$ are
  not in $L^2(\Omega)$, we only have $I_hu\in H^1(\Omega)$.
  }
\EndParts
\end{Question}

\label{ptotal}
\EndPage
\end{document}
