\documentclass[12pt]{article}

\usepackage{graphicx,amssymb,amsmath,nicefrac,ifthen}
\usepackage[T1]{fontenc}
\usepackage{lmodern}

%%%%%%%%%%%%%%%%%%%%%%%%%%%%%%%%%%%
%  Fill in the following details  %
%%%%%%%%%%%%%%%%%%%%%%%%%%%%%%%%%%%

\newcommand{\coursenum}{M4MA47} %{e.g: M1S, or: M3S14/M4S14/M5S15 for multiple year courses}
\newcommand{\coursename}{Finite elements: numerical analysis and implementation} %{e.g: Probability and Statistics} 
\newcommand{\examyear}{2017}
\newcommand{\exammonth}{May -- June} 
\newcommand{\examdate}{??} %{Tuesday, 26th May 2017} 
\newcommand{\examtime}{??} %{09:30 -- 11:30} 
\newcommand{\examlength}{??} %{2} length of exam in hours
\newcommand{\numQ}{??} %{Four} number of questions
\newcommand{\setter}{Colin Cotter}     % Insert surname of exam setter 
\newcommand{\checker}{David Ham} % Insert surname of exam checker 
\newcommand{\editor}{Andrew Walton} % Insert surname of exam editor
\newcommand{\external}{external} % Insert surname of external examiner
\newboolean{tables}
\setboolean{tables}{false} %set to true if statistical tables are provided
\newboolean{onebook}
\setboolean{onebook}{false} %set to true if all answers should be in one answer book

%%%%%%%%%%%%%%%%%%%%%%%%%%%%%%%%

\pretolerance=100000
\setlength{\topmargin}{-15mm}
\setlength{\textheight}{245mm}
\setlength{\textwidth}{178mm}
\setlength{\oddsidemargin}{-10mm}
\setlength{\evensidemargin}{-10mm}
\setlength{\marginparwidth}{1cm}
\setlength{\parskip}{1.1ex}
\setlength{\parindent}{0ex}
\renewcommand{\baselinestretch}{1.1}

\pagestyle{empty}

\newenvironment{Question}[1] 
 {\begin{itemize} \item[\large #1.~~]}{\end{itemize} \medskip}
 
\newcommand{\EndPage}{
	\vfill \coursenum ~ \coursename ~
	(\examyear) \hfill Page \thepage \newpage
	}

\newcommand{\ICLOGO}{
	\begin{minipage}{0.7\textwidth}
  \includegraphics[height= 1.4cm]{imperial.pdf} 
  \end{minipage}\hfill \parbox[r][1.4cm][t]{0.2\textwidth}{\hfill \coursenum} \par 
	}
	
\newcommand{\BeginParts}{\begin{itemize}} 
\newcommand{\Part}[1]{\item [(#1)~~]} 
\newcommand{\EndParts}{\end{itemize}} 

%%%%
% This requires imperial.pdf in the graphics input path eg 
% h:/images/imperial.pdf 
%%%% 


\newcommand{\fpagedraft}{  % FOR DRAFT FRONT PAGE with SIGS
	\ICLOGO
	\begin{center} 
	\draft
	BSc, MSci and MSc EXAMINATIONS (MATHEMATICS) \par 
	\exammonth~ \examyear \vfill \par
	\Large \coursenum \bigskip \par
	\coursename \bigskip \par
	\end{center} \vfill \par \normalsize
	\sigs  \par \vfill 
	\copyright ~\examyear~ Imperial College London
	\hfill \coursenum \hfill draft cover 
	\newpage
	}

\newcommand{\draft}{
	\begin{flushright} \begin{tabular}{ll}
  Course:   & \coursenum\\ Setter:   & \setter \\
  Checker:  & \checker \\  Editor:   & \editor \\
  External: & \external \\ Date:     & \today
  \end{tabular} \end{flushright} \vfill \par
	}
	
\newcommand{\sigs}{ % Signatures
	\vfill \par
  \fbox{\begin{minipage}{0.98\textwidth}{~ \\[4mm]
  \hspace*{3mm} Setter's signature \hfill
  Checker's signature \hfill
  Editor's signature~~~~ \\[4mm]
  \hspace*{3mm} \dotfill \hfill \dotfill \hfill \dotfill ~~~~
  \\[2mm] ~} \end{minipage}}
	}
	
	
\newcommand{\fpage}{ % FRONT PAGE 
	\ICLOGO
  \begin{center}
  BSc, MSc and MSci EXAMINATIONS (MATHEMATICS) \par
  \exammonth~ \examyear \bigskip \par
  This paper is also taken for the relevant examination
  for the Associateship of the \\ Royal College of Science. 
	\end{center}
	\begin{center}
	\coursename
	\end{center}
	\renewcommand{\arraystretch}{1.3}
	\vfill \par
	\fbox{\parbox{\textwidth}{
	\begin{tabular}{l}
	Date: \examdate \\ 
	Time: \examtime \\
	Time Allowed: \examlength~Hours \\
	This paper has {\em \numQ} Questions. \\
	\ifthenelse{\boolean{onebook}}
	{Candidates should use ONE main answer book.}
	{Candidates should start their solutions to each question in a new main answer book.}\\
	%Candidates should use ONE main answer book.\\
	Supplementary books may only be used after the relevant main book(s) are full. \\
	\ifthenelse{\boolean{tables}}{Statistical tables are provided.}{Statistical tables will not be provided.}\\
	\end{tabular}}}
	\renewcommand{\arraystretch}{1.5}
	\begin{itemize}
	\item DO NOT OPEN THIS PAPER UNTIL THE INVIGILATOR TELLS YOU TO.
	\item Affix one of the labels provided to each answer book that you use, 
	but DO NOT USE THE LABEL WITH YOUR NAME ON IT.
	\item	Credit will be given for all questions attempted, but extra credit will be given for complete or
	nearly complete answers to each question as per the table below.
	\begin{center}
	\begin{tabular}{|l|c|c|c|c|c|c|c|c|c|} \hline
	Raw Mark & Up to 12 & 13&14&15&16&17&18&19&20 \\ \hline 
	Extra Credit & 0 & $\nicefrac{1}{2}$ & 1 & $1 \nicefrac{1}{2}$ & 2 & $2 \nicefrac{1}{2}$ & 3 & 
	$3 \nicefrac{1}{2}$ & 4 \\ \hline
	\end{tabular}
	\end{center}
	\item Each question carries equal weight.
	\item Calculators may not be used.
	\end{itemize}
	\vfill 
	\copyright ~\examyear~ Imperial College London
	\hfill \coursenum \hfill Page 1 of \pageref{ptotal}
	\newpage
	}

% Macros for this exam
\DeclareMathOperator{\diff}{d}
\newcommand{\pp}[2]{\frac{\partial #1}{\partial #2}} 
%uncomment \solnsfalse to remove solution set
\newif\ifsolns
%\solnstrue
\solnsfalse
\ifsolns
% with solutions
\usepackage{color}
\newcommand{\soln}[1]{\newline \noindent {\bfseries Solution:} {\itshape \color{blue} #1}}
\else
% without solutions
\newcommand{\soln}[1]{}
\fi
% Exam text starts from here
%
% use \begin{Question}{1} ... \end{Question} and
%     \BeginParts \Part{a} \EndParts
%
% manually put in \EndPage at the end of every page for the correct footers.

\begin{document}
\sffamily
\fpagedraft % this produces the signature page 
\setcounter{page}{1}
\fpage % this produces the front page with rubric

\begin{Question}{1}
\BeginParts
\Part{a} Let $(K,\mathcal{P},\mathcal{N})$ be such that:
\begin{enumerate}
\item $K$ is a (non-degenerate) triangle,
\item $\mathcal{P}$ is the space of polynomials of degree $\leq 1$.
\item $\mathcal{N}$ is the set of nodal variables with
\[
N_1(u) = \int_K u \diff x, \quad 
N_2(u) = \pp{u}{x}, \quad N_3(u) = \pp{u}{y}.
\]
\end{enumerate}
Find the nodal basis for $\mathcal{P}$ corresponding to $\mathcal{P}$
by expanding in the monomial basis
\[
\psi_1(x) = 1, \quad \psi_2(x) = x, \quad \psi_3(x) = y. 
\]
(Hint: to make calculation easier, you may make use of the fact that
matrices of the form
\[
\begin{pmatrix}
a & 0 & 0 \\
b & 1 & 0 \\
c & 0 & 1 \\
\end{pmatrix}
\]
form a group.)
\soln{
We expand $\phi_i(x)=\sum_{j=1}^3 a_{ij}\psi_j(x)$. Then,
\[
\delta_{ik} = N_k(\phi_i) = \sum_{j=1}^3 a_{ij}N_k(\psi_j) =
(AV)_{ik},
\]
where $A$ is the matrix with coefficients $a_{ij}$, and
$V$ is the matrix with coefficients $v_{ij}=N_j(\psi_i)$, given by
\[
V = \begin{pmatrix}
  |K| & 0 & 0 \\
  \bar{x}|K| & 1 & 0 \\
  \bar{y}|K| & 0 & 1 \\
\end{pmatrix},
\quad |K|=\int_K\diff x, \, \bar{f} = \frac{\int_K f(x)\diff x}{|K|}.
\]
Using the hint, we write
\[
V^{-1} = 
\begin{pmatrix}
a & 0 & 0 \\
b & 1 & 0 \\
c & 0 & 1 \\
\end{pmatrix},
\]
so
\[
V^{-1}V = 
\begin{pmatrix}
a|K| & 0 & 0 \\
|K|(b + \bar{x}) & 1 & 0 \\
|K|(c + \bar{y}) & 0 & 1 \\
\end{pmatrix}
=
\begin{pmatrix}
1 & 0 & 0 \\
0 & 1 & 0 \\
0 & 0 & 1 \\
\end{pmatrix}
\implies 
V^{-1}
=
\begin{pmatrix}
\frac{1}{|K|} & 0 & 0 \\
-\bar{x} & 1 & 0 \\
-\bar{y} & 0 & 1 \\
\end{pmatrix},
\]
hence we obtain the basis
\[
\phi_1(x) = \frac{1}{|K|}, \, \phi_2(x) = x - \bar{x},
\phi_3(x) = y - \bar{y}.
\]
}
\Part{b} Use this calculation to explain why $\mathcal{N}$ determines
$\mathcal{P}$.  \soln{The functions $\{\phi_i\}_{i=1}^3$ satisfies the
  requirements of the dual basis and are written as an invertible
  transformation from the monomial basis, hence they are linearly
  independent and span $\mathcal{P}$.}  
\Part{c} Comment on the relative sizes of the basis functions as the triangle area goes to zero, and suggest a rescaling of the nodal variables that
remedies this.
\soln{
As $|K|$ goes to zero, $\phi_1\to 0$ whilst $\phi_2$ and $\phi_3$ stay finite.
This can be fixed by replacing $N_1$ with $\tilde{N}_1=\bar{u}$, in which
case $V$ becomes
\[
V = \begin{pmatrix}
  1 & 0 & 0 \\
  \bar{x} & 1 & 0 \\
  \bar{y} & 0 & 1 \\
\end{pmatrix},
\]
so
\[
\psi_1(x) = 1, \, \psi_2(x) = x-\bar{x}, \, \psi_3(x) = y-\bar{y}.
\]
}
\EndParts
\end{Question}
\EndPage
\begin{Question}{2} 
We consider the finite element $(K,\mathcal{P},\mathcal{N})$ where
\begin{enumerate}
\item $K$ is a (non-degenerate) triangle,
\item $\mathcal{P}$ is the space $(P_1)^2$ of vector-valued polynomials
(i.e. each vector component is in $P_1$).
\item Elements of $\mathcal{N}$ are dual functions that return the
  normal component of vector fields at the end of each edge (2
  evaluations per edge, one at each end, and 3 edges, makes 6 dual
  functions in total).
\end{enumerate}
The geometric decomposition of $(K,\mathcal{P},\mathcal{N})$ is defined
by associating each dual basis function with the edge where the normal
component is evaluated.\\
We consider the finite element space $V$ defined on a triangulation
$\mathcal{T}$ of a polygonal domain $\Omega$, constructed from the
element above, so that dual basis evaluations agree for triangles
on either side of each interior edge.
\BeginParts
\Part{a} Show that the weak divergence $\nabla_w\cdot u$ exists for
$u\in V$, defined by
\[
\int_\Omega \phi \nabla_w\cdot u \diff x = -\int_\Omega \nabla \phi\cdot u 
\diff x, \quad \forall \phi \in C_0^\infty(\Omega).
\]
\soln{
First we note that the normal components of $u\in V$ agree on both
sides of each interior edge, since $u.n$ restricted to the edge is 
a linear scalar function. Since the values of $u.n$ on each side agree
at two points, the difference $u^+.n^+ + u^-\cdot n^-$ is a linear function
that vanishes at two points, and hence is zero.

We define
\[
\nabla_w\cdot u|_K = \nabla\cdot u|_K,
\]
for all triangles $K\in\mathcal{T}$. Then 
\begin{align*}
\int_\Omega \phi \nabla_w \cdot u \diff x & = \sum_K \int_K \phi \nabla\cdot u\diff x, \\
& = \sum_K \left(-\int_K \nabla\phi \cdot u\diff x + \int_{\partial K}
\phi n\cdot u \diff S\right), \\
& = -\int_\Omega \nabla\phi \cdot u \diff x,
\end{align*}
as required, since the normal components agree.
}
\Part{b} Develop a variational formulation for the problem
\[
u - c\nabla(\nabla\cdot u) = f, \mbox{ for }x\in \Omega,
\quad u.n=0 \mbox{ on }\partial\Omega,
\]
using the finite element space $V$, for $c$ a positive
constant. Develop an inner product that gives coercivity and continuity for the
corresponding bilinear form with respect to the corresponding normed space.
\soln{
Taking inner product with test function $w$, integration by parts and 
application of the boundary condition gives
\[
a(u,v) = F(v), \, \forall v\in V, \quad a(u,v) = \int_\Omega u.v 
+ c\nabla\cdot u\nabla\cdot v \diff x, \quad F(v)=\int_\Omega f\cdot v 
\diff x.
\]
$a(u,v)$ is symmetric, bilinear, positive-definite in $L^2$, thus we 
may use it as an inner product. $a$ therefore has continuity and coercivity
constants equal to 1 with respect to the corresponding norm.
}
\Part{c} Show that $u\in V$ does not have a weak curl $\nabla^\perp_w \cdot u$
in general, where
\[
\int_\Omega \Phi \cdot \nabla^\perp_w \cdot u \diff x = \int_\Omega \nabla^\perp
\Phi \cdot u \diff x, \quad \forall \Phi \in C_0^\infty(\Omega),
\]
where $\nabla^\perp=(-\pp{}{y},\pp{}{x})$. [Hint: show this by
  counter-example.  First choose a function $u\in V$ that you think
  does not have a weak curl. Then consider a suitable limit of smooth
  test functions that contradicts the above definition of a weak
  curl.]  \soln{ Let $f_1$ be an interior edge joining two triangles
  $e_1$ and $e_2$, and let $f_2$ be one of the other edges of
  $e_2$. Take $u.n=1$ on $f_2$, and $u.n=0$ on all other edges. Then,
  on the $e_1$ side of $f_1$, $u.t=0$, where $t$ is the unit tangent
  to $f_1$, oriented so that $n_1^\perp=t$, where $n_1$ is the unit
  normal pointing from $e_1$ to $e_2$. On the $e_2$ side, $u.t\neq 0$
  (because otherwise $u.n=0$ at the $f_1$ end of $f_2$).  Define
  $g=u.t|_{e_2}$ on $f_1$.  Then, pick a sequence of $C^\infty_0$ test
  functions $\phi_k$ such that
\[
\phi_k|_{f_2}  \to 1, \, \phi_k|_{e_1} \to 0\, \phi_k|_{e_2} \to 0,
\]
and $\phi_k \to 0$ in all triangles other than $e_1$, $e_2$. We have
\begin{align*}
\int_\Omega \nabla^\perp \Phi \cdot u \diff x & = 
\sum_K\int_K \nabla^\perp \Phi \cdot u \diff x, \\
& = \sum_K\left(-\int_K \nabla^\perp \Phi \cdot u \diff x
+ \int_{\partial K} \Phi n^\perp \cdot u \diff S\right), \\
& \to \int_f g \Phi \diff S \neq 0.
\end{align*}
On the other hand, if $\nabla^\perp_wu$ exists, then
\[
0 \neq \int_f g \Phi \diff S = \int_\Omega \Phi \nabla^\perp_w\cdot u \diff S
= \sum_K \int_\Omega \Phi|_K \nabla^\perp_w \cdot u \diff S \to 0,
\]
which is a contradiction.
}
\EndParts
\end{Question}
\EndPage

\begin{Question}{3}
  We consider the following boundary value problem in one dimension.
  \[
  -u'' + (2+\sin(x))u = f(x), \quad u(0) = 0, \, u'(1)=1.
  \]
    \BeginParts
    \Part{a} Construct a formulation of this problem describing
    a weak solution $u$ in $H^1([0,1])$.
    \soln{
      Multiplication by test function $v$ satisfying $v(0)=0$, and integration by parts, using the boundary condition gives
      \[
      \int_0^1 v'u' + (2+\sin(x))vu \diff x = \int_0^1 vf \diff x
      + v(1),
      \quad \forall v\in H^1_0([0,1]),
      \]
      where $H^1_0([0,1])$ is the subspace of $H^1([0,1])$ such that
      $v(0)$.
      }
    \Part{b} Show that the corresponding bilinear form is continuous
    and coercive in $H^1([0,1])$, and compute the continuity and
    coercivity constants.
    \soln{
      Continuity:
      \begin{align*}
        a(u,v) & \leq \|v'\|_{L^2(\Omega)}\|u'\|_{L^2(\Omega)}
        + 3\|v\|_{L^2(\Omega)}\|u\|_{L^2(\Omega)}, \\
        & \leq 3\|u\|_{H^1(\Omega)}\|v\|_{H^1(\Omega)}.
      \end{align*}
      Constant is 3.\\
      Coercivity:
      \begin{align*}
        a(u,u) & = \int_0^1 (u')^2 + (2+\sin(x))u^2 \diff x, \\
        & \geq \int_0^1 (u')^2 + u^2 \diff x = \|u\|^2_{H^1(\Omega)}.
      \end{align*}
      Constant is 1.
    }
    \Part{c} What is the required property of $f$ for a unique solution
    $u$ to exist?
    \soln{
      The RHS functional is
      \[
      L[v] = \int_0^1 fv \diff x + v(1).
      \]
      We have
      \[
      |L[v]| \leq \|f\|_{L^2([0,1])}\|v\|_{L^2([0,1])} + C\|v\|_{H^1([0,1])}
      \leq (\|f\|_{L^2([0,1])} + C)\|v\|_{H^1([0,1])},
      \]
      where $C$ is the trace constant for $H^1([0,1])$. Hence we need
      $f\in L^2([0,1])$ for $L$ to be bounded/continuous and hence
      the conditions of Lax-Migram to be satisfied.
      }
    \Part{d} Describe the piecewise linear $C^0$ finite element discretisation
    of this equation with mesh vertices $[x_0=0,x_1,x_2,\ldots,x_n,x_{n+1}=1]$.
    \soln{
      Let $V_h$ be the finite element space of continuous, piecewise linear
      functions defined on this mesh, with subspace $\mathring{V}_h$ containing
      only the functions that satisfy $v_h(0)=0$. The Galerkin finite
      element discretisation seeks $u_h\in \mathring{V}_h$ such that
      \[
      a(u_h,v) = L[v], \quad \forall v\in \mathring{V}_h.
      \]      
      }
    \Part{e} Given an arbitrary basis of the finite element space $V_h$,
    show that the resulting matrix $A$ is symmetric ($A^T=A$) and positive
    definite, i.e. $x^TAx > 0$ for all $x$ with $\|x\|>0$.
    \soln{
      Given a basis $\{\phi_i\}_{i=1}^n$, $A$ is defined by
      \[
      A_{ij} = \int_0^1 \phi'_i\phi'_j + (2+\sin(x))\phi_i\phi_j\diff x.
      \]
      $A$ is symmetric, since exchanging $i$ and $j$ returns the same answer.\\
      $A$ is PD since if we take
      \[
      u = \sum_{i=1}^n x_i \phi_i,
      \]
      then
      \begin{align*}
      x^TAx &= \sum_{ij}x_iA_{ij}x_j
      = \int_0^1 \sum_ix_i\phi'_i\sum_j x_j\phi'_j +
      (2+\sin(x))\sum_ix_i\phi_i\sum_j x_j\phi_j \diff x, \\
      & = \int_0^1 (u')^2 + (2+sin(x))u^2 \diff x,
      \end{align*}
      which is positive due to the coercivity result.
      }
    \Part{f} Show that the numerical solution $u_h$ satisfies
    $\|u-u_h\|_{H^1([0,1])}=\mathcal{O}(h)$ as $h\to 0$. [You may
      quote any properties of the interpolation operator
      $\mathcal{I}_h$ without proof, but must show the other steps.]
    \soln{ Since $\mathring{V}_h\in H^1([0,1])$, we have
      \[
      a(v,u-u_h) = 0, \quad \forall v \in \mathring{V}_h.
      \]
      Then, taking $v\in \mathring{V}_h$,
      \begin{align*}
        \|u-u_h\|_{H^1([0,1])}^2 & \leq a(u-u_h,u-u_h), \\
        & = \underbrace{a(u-u_h,v-u_h)}_{=0} + a(u-u_h,u-v), \\
        & \leq 3\|u-u_h\|_{H^1}\|u-v\|_{H^1}.
      \end{align*}
      Dividing by $\|u-u_h\|_{H^1}$ and taking $v=\mathcal{I}_hu$ gives
      \[
      \|u-u_h\|_{H^1} \leq 3\|u-\mathcal{I}_hu\|
      \leq 3h\|u\|_{H^1},
      \]
      where we quoted the property of $\mathcal{I}_h$ in the last inequality.
    }
    \EndParts
\end{Question}

\EndPage

%% \begin{Question}{3}
%% The scalar advection diffusion equation
%% \[
%% b.\nabla u = \epsilon \nabla^2 u + f, \quad \mbox{ for }x\in \Omega,
%% \quad u=0 \mbox{ on }\partial\Omega,
%% \]
%% for constant $\epsilon>0$, and a divergence-free vector field $b$
%% satisfying $b\cdot n=0$ on $\partial \Omega$,
%% has variational form
%% \[
%% \int_\Omega v b\cdot\nabla u + \epsilon \nabla v \cdot \nabla u \diff x
%% = \int_\Omega f v \diff x,
%% \quad \forall v \in H^1(\Omega),
%% \]
%% for $u \in H^1(\Omega)$.
%% \BeginParts
%% \Part{a} Determine continuity and coercivity constants for the corresponding 
%% bilinear form; under what conditions for $b$ are they finite?
%% You may make use of the Poincar\'e estimate
%% \[
%% \int_\Omega u^2 \diff x \leq C_\Omega \int_\Omega |\nabla u|^2 \diff x.
%% \]
%% \soln{
%% For continuity we have
%% \begin{align*}
%% \int_\Omega v b\cdot\nabla u + \epsilon \nabla v \cdot \nabla u \diff x
%%  & \leq \|v\|_{L^2(\Omega)}\|\nabla u\|_{L^2(\Omega)}\|b\|_{L^\infty(\Omega)}
%%  + \epsilon \|\nabla v\|_{L^2(\Omega)}\|\nabla u\|_{L^2{\Omega}}, \\
%%  & \leq (\|b\|_{L^\infty(\Omega)}+\epsilon)\|v\|_{H^1(\Omega)}\|u\|_{H^1(\Omega)},
%% \end{align*}
%% so the continuity constant $M=(\|b\|_{L^\infty(\Omega)}+\epsilon)$. Hence
%% we require $\|b\|_{L^\infty(\Omega)}<\infty$. \\
%% For coercivity, note that
%% \[
%% \int_\Omega u b\cdot \nabla u \diff x = \int_\Omega b\cdot \nabla
%% \frac{1}{2}u^2 \diff x = -\int_\Omega \frac{1}{2}u^2\nabla\cdot b\diff x
%% = 0.
%% \]
%% Thus,
%% \[
%% a(u,u) = \int_{\Omega} \epsilon |\nabla u|^2 \diff x.
%% \]
%% The Poincar\'e inequality gives
%% \[
%% \|u\|_{H^1(\Omega)}^2=\|u\|^2_{L^2(\Omega)} + \|\nabla u\|^2_{L^2(\Omega)} \leq \frac{C^2+1}{\epsilon}a(u,u),
%% \]
%% hence the coercivity constant $\gamma = \sqrt{\epsilon/(C^2+1)}$.
%% }
%% \Part{b} The advection diffusion equation is discretised using continuous 
%% finite element methods. Using C\'ea's Theorem, comment on the bounds on
%% the error as $\epsilon \to 0$. 
%% \soln{
%% C\'ea's Theorem says that 
%% \[
%% \|u-u_h\|_V \leq \frac{M}{\gamma}\min_{v\in V_h}\|u-v\|_V.
%% \]
%% The constant $M/\gamma\to \infty$ as $\epsilon\to 0$, and hence
%% C\'ea's Theorem provides no bounds on the error in this limit.  }
%% \EndParts
%% \end{Question} 

\begin{Question}{4}
Consider the finite element $(K,\mathcal{P},\mathcal{N})$, with 
\begin{enumerate}
\item $K$ is a non-degenerate triangle,
\item $\mathcal{P}$ is the space of polynomials on $K$ of degree $\leq 2$.
\item $\mathcal{N}=(N_{1,1},N_{1,2},N_{2,1},N_{2,2},N_{3,1},N_{3,2})$,
where 
\[
N_{i,j}(u)= \int_{f_i}\phi_{i,j} u\diff x,
\]
where $(f_1,f_2,f_3)$ are the edges of $K$, with $f_1$ joining
vertices $1$ and $2$, $f_2$ joining vertices $2$ and $3$, and $f_3$
joining vertices 3 and 1. The edge test functions $\phi_{i,j}$ define
a basis for linear functions restricted to $f_i$ such that $\phi_{1,1}=1$
on vertex $1$ and 0 on vertex $2$, \emph{etc.}
 \BeginParts
 \Part{a} Show that $\mathcal{N}$ determines $\mathcal{P}$. 
 \soln{It suffices to show that if $u \in P$, then
 $N_{i,j}(u)=0$ for all $i,j$ $\implies u=0$. Under this assumption,
 we may expand $u|_{f_1} = u_1\phi_{1,1}+u_2\phi_{1,2}$, hence
 \[
 \int_{f_i}u^2 \diff x = 0, \implies u|_{f_1}=0.
 \]
Similarly we obtain $u|_{f_2}=0$, $u|_{f_3}=0$.
If $u|_{f_1}=0$, then $u(x)=L_1(x)Q_1(x)$ where $L_1$ is a non-degenerate
 linear polynomial with $L_1|_{f_1}=0$.
Since $L_1|_{f_2}\ne 0$, we
 conclude that $Q_1(x)|_{f_2}=0$, so that $Q_1(x)=cL_2(x)$, where $c$
 is a constant, and $L_2|_{f_2}=0$. Finally since $L_2|_{f_3}\ne 0$, 
 we conclude that $c=0$, hence the result.}
 \Part{b} We take a geometric decomposition such that $N_{i,j}$ is 
associated with $f_i$, $i=1,2,3$, $j=1,2$. What is the continuity of the
corresponding finite element space $V$ defined on a triangulation $\mathcal{T}$
of a polygonal domain $\Omega$? Explain your answer.
\soln{ There exist discontinuous functions in $V$, so functions are
  not even in $C^0(\Omega)$. To see this, pick two neighbouring
  triangles $e_1$, $e_2$ with common face $f_1$, and another face
$f_2$ in $e_1$. We pick $u=1$ on $f_2$ and $u=0$ when restricted to 
all other faces. This function is equal to zero on the $e_1$ side of $f_1$,
and is a linear function going from 1 to 0 on the $e_2$ side, hence
discontinuous.
}
\Part{c} We assume that $u\in H^k(\Omega)$ for some non-negative integer $k$.
What is the minimum value of $k$ such that the global interpolator
$\mathcal{I}_hu$ is well-defined? Explain your answer.
\soln{
We require that $u|_f \in L^2(f)$ for each edge in $\mathcal{T}$. This
is guaranteed if $u \in H^1(\Omega)$ via the trace theorem. On the other
hand $u\in L^2(\Omega)$ is not enough to guarantee that $u|_f\in L^2(f)$.
So $k=1$.
}
 \EndParts
\end{enumerate}

\end{Question}

%% If you have a Mastery Question, then this should be included as an additional question 
%% on a new page, uncomment the following three lines:
% \EndPage
% \begin{Question}{5}
% \end{Question}


\label{ptotal}
\EndPage
\end{document}
