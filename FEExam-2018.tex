
\documentclass[12pt]{article}

\usepackage{graphicx,amssymb,amsmath,nicefrac,ifthen,calc}
\usepackage[T1]{fontenc}
\usepackage{lmodern}

%%%%%%%%%%%%%%%%%%%%%%%%%%%%%%%%%%%
%  Fill in the following details  %
%%%%%%%%%%%%%%%%%%%%%%%%%%%%%%%%%%%

\newcommand{\coursenum}{M4A47/M5A47} %{e.g: M1S, or: M3S14/M4S14/M5S15 for multiple year courses}
\newcommand{\coursename}{Finite elements: numerical analysis and implementation} %{e.g: Probability and Statistics} 
\newcommand{\setter}{Cotter}     % Insert surname of exam setter 
\newcommand{\checker}{Ham} % Insert surname of exam checker 
\newcommand{\editor}{Walton} % Insert surname of exam editor
\newcommand{\external}{external} % Insert surname of external examiner
\newboolean{tables}
\setboolean{tables}{false} %set to true if statistical tables are provided and attach require tables

% Choose an option for time allowed and number of questions %

\newcommand{\examlength}%Uncomment one of the following 3 options, or adjust.
{2~Hours}

\newcommand{\numQ}%Uncomment one of the following 3 options, or adjust.
{4~Questions}

\newcommand{\version}% Uncomment one of the following, or adjust
{Draft version for checking}
%{Version for External Examiner}
%{Final version}


\newcommand{\suitability}% Uncomment one of the options below
{Is the paper suitable for resitting students from previous years:
Yes
}



%%%%%%%%%%%%%%%%%%%%%%%
%The following details about the difficulty level of the questions must be completed 
%%%%%%%%%%%%%%%%%%%%%%%

\newcommand{\routine}% Indicate, with marks, the most straightforward 40 percent of the marks available in the non-mastery questions
{1(a,b,c) 20 marks; 3(a,b) 8 marks; 4(a) 6 marks.
}
\newcommand{\medium}% Indicate the next most challenging 25 percent of the marks in the non-mastery questions
{2(a) 10 marks, 3(c) 6 marks, 4(b) 6 marks
}

\newcommand{\hardest}% Indicate the most challenging 20 percent of marks on the paper in the non-mastery questions
{2(b) 10 marks, 3(d) 6 marks, 4(c) 8 marks
}



%%%%%%%%%%%%%%%%%%%%%%%%%%%%%%%%%%%%%
% Insert your macros %
%%%%%%%%%%%%%%%%%%%%%%%%%%%%%%%%%%%%%

%uncomment \solnsfalse to remove solution set
\newif\ifsolns
\solnstrue
%\solnsfalse
\ifsolns
% with solutions
\usepackage{draftwatermark}
\SetWatermarkText{Solutions}
\SetWatermarkScale{0.6}
\SetWatermarkLightness{0.9}
\usepackage{color}
\newcommand{\soln}[1]{{\bfseries Solution:} {\itshape \color{blue} #1}}
\else
% without solutions
\newcommand{\soln}[1]{}
\fi

\newcommand{\exammarks}[1]{\begin{flushright}[#1 marks]\end{flushright}}%
\DeclareMathOperator{\diff}{d}
\newcommand{\pp}[2]{\frac{\partial #1}{\partial #2}} 
\newcommand{\bookwork}{{\bfseries BOOKWORK\\}}
\newcommand{\similar}{{\bfseries SIMILAR\\}}
\newcommand{\unseen}{{\bfseries UNSEEN\\}}

% Go to `Exam text starts from here' below to type in your questions. %

%%%%%%%%%%%%%%%%%%%%%%%%%%%%%%%%%%%%%%%%%%%%%%%%%%%%%%%%%%%%%%%%%
%%%%%%%%%%%%%%%%%%%%%%%%%%%%%%%%%%%%
% Ignore the following details     %
%%%%%%%%%%%%%%%%%%%%%%%%%%%%%%%%%%%%

\newcommand{\examyear}{2018}
\newcommand{\exammonth}{May -- June} 
\newcommand{\examdate}{??} %{Tuesday, 26th May 2018} 
\newcommand{\examtime}{??} %{09:30 -- 11:30} 
\newboolean{onebook}
\setboolean{onebook}{false} %set to true if all answers should be in one answer book

%%%%%%%%%%%%%%%%%%%%%%%%%%%%%%%%%%%%

\pretolerance=100000
\setlength{\topmargin}{-15mm}
\setlength{\textheight}{245mm}
\setlength{\textwidth}{178mm}
\setlength{\oddsidemargin}{-10mm}
\setlength{\evensidemargin}{-10mm}
\setlength{\marginparwidth}{1cm}
\setlength{\parskip}{1.1ex}
\setlength{\parindent}{0ex}
\renewcommand{\baselinestretch}{1.1}

\pagestyle{empty}

\newenvironment{Question}[1] 
 {\begin{itemize} \item[\large #1.~~]}{\end{itemize} \medskip}
 
\newcommand{\EndPage}{
	\vfill \coursenum ~ \coursename ~
	(\examyear) \hfill Page \thepage \newpage
	}

%\newcommand{\ICLOGO}{
%	\begin{minipage}{0.7\textwidth}
%  \includegraphics[height= 1.4cm]{imperial.pdf} 
%  \end{minipage}\hfill \parbox[r][1.4cm][t]{0.2\textwidth}{\hfill \coursenum} \par 
%	}
	
\newcommand{\BeginParts}{\begin{itemize}} 
\newcommand{\Part}[1]{\item [(#1)~~]} 
\newcommand{\EndParts}{\end{itemize}} 

%%%%
% This requires imperial.pdf in the graphics input path eg 
% h:/images/imperial.pdf 
%%%% 

\newcommand{\draft}{
	\begin{flushleft} 
	\begin{tabular}{ll}
  Module:   & \coursenum\\ Setter:   & \setter \\
  Checker:  & \checker \\  Editor:   & \editor \\
  External: & \external \\ Date:     & \today\\
  Version: & \version
  \end{tabular} 
  \end{flushleft} 
  \vfill \par
	}


\newcommand{\fpagedraft}{  % FOR DRAFT FRONT PAGE with SIGS
%	\ICLOGO
	\begin{center} 
	\draft
	BSc, MSci and MSc EXAMINATIONS (MATHEMATICS) \par 
	\exammonth~ \examyear 
	
	\medskip
	
	\large \coursenum \quad
	\coursename 
	\end{center}
	\medskip
	
	\textit{The following information must be completed:}
	
	\medskip
	
	\textbf{\suitability}
	
	\medskip
	
	\textbf{Marks available for basic, routine material (excluding any mastery question)(40 percent = 32/80 for 4 questions):}\newline
	\routine
	
	\medskip
	
	\textbf{Further 25 percent of marks (20/ 80 for  4 questions) for demonstration of a sound knowledge of a 
	good part of the material and the solution of straightforward problems and examples with reasonable accuracy (excluding mastery question):} \newline
	\medium
	\medskip
	
	\textbf{Most challenging 20 percent (16/80 marks for 4 questions) of the paper (excluding mastery question):}\newline
	\hardest
	
	
	
	
	 \vfill \par \normalsize
	 
	
	
	\textit{Signatures are required for the final version:}
		\sigs  \par \vfill 
	\copyright ~\examyear~ Imperial College London
	\hfill \coursenum \hfill Temporary cover page
	\newpage
	}


	
	
\newcommand{\sigs}{ % Signatures
	\vfill \par 
  \fbox{\begin{minipage}{0.98\textwidth}{~ \\[4mm]
  \hspace*{3mm} Setter's signature \hfill
  Checker's signature \hfill
  Editor's signature~~~~ \\[4mm]
  \hspace*{3mm} \dotfill \hfill \dotfill \hfill \dotfill ~~~~
  \\[2mm] ~} \end{minipage}}
	}
	


	
\newcommand{\fpage}{ % FRONT PAGE 
%	\ICLOGO
%  
\newlength{\myl}
\settowidth{\myl}{\sc Temporary front page}
\newlength{\myll}
\setlength{\myll}{\textwidth - \myl}
\begin{center}
 \large{\sc \rule[4pt]{0.45\myll}{0.5pt}~Temporary front page~\rule[4pt]{0.45\myll}{0.5pt}}
 \end{center}
 %
 \bigskip
  \begin{center}
  BSc, MSc and MSci EXAMINATIONS (MATHEMATICS) \par
  \exammonth~ \examyear \bigskip \par
  This paper is also taken for the relevant examination
  for the Associateship of the \\ Royal College of Science. 
	\end{center}
	\begin{center}
	\coursename
	\end{center}
	\renewcommand{\arraystretch}{1.3}
	\vfill \par
	\fbox{\parbox{\textwidth}{
	\begin{tabular}{l}
	Date: \examdate \\ 
	Time: \examtime \\
	Time Allowed: \examlength \\
	This paper has {\em \numQ}. \\
	\ifthenelse{\boolean{onebook}}
	{Candidates should use ONE main answer book.}
	{Candidates should start their solutions to each question in a new main answer book.}\\
	%Candidates should use ONE main answer book.\\
	Supplementary books may only be used after the relevant main book(s) are full. \\
	\ifthenelse{\boolean{tables}}{Statistical tables are provided.}{Statistical tables will not be provided.}\\
	\end{tabular}}}
	\renewcommand{\arraystretch}{1.5}
	\begin{itemize}
	\item DO NOT OPEN THIS PAPER UNTIL THE INVIGILATOR TELLS YOU TO.
	\item Affix one of the labels provided to each answer book that you use, 
	but DO NOT USE THE LABEL WITH YOUR NAME ON IT.
	\item	Credit will be given for all questions attempted.
	\item Each question carries equal weight.
	\item Calculators may not be used.
	\end{itemize}
	\vfill 
	\copyright ~\examyear~ Imperial College London
	\hfill \coursenum \hfill Page 1 of \pageref{ptotal}
	\newpage
	}
%%%%%%%%%%%%%%%%%%%%%%%%%%%%%%%%%%%%%%%%%%%%%%%%%%%%%%%%%%%%%%%%
%%%%%%%%%%%%%%%%%%%%%%%%%%%%%%%%%%%%%%%%%%%%%%%%%%%%%%%%%%%%%%%%
% Exam text starts from here
%%%%%%%%%%%%%%%%%%%%%%%%%%%%%%%%%%%%%%%%%%%%%%%%%%%%%%%%%%%%%%%%%
%%%%%%%%%%%%%%%%%%%%%%%%%%%%%%%%%%%%%%%%%%%%%%%%%%%%%%%%%%%%%%%%%

% use \begin{Question}{1} ... \end{Question} and
%     \BeginParts \Part{a} \EndParts
%
% manually put in \EndPage at the end of every page for the correct footers.

\begin{document}
\sffamily
%\fpagedraft % this produces the signature page 
\setcounter{page}{1}
\fpage % this produces the front page with rubric

\begin{Question}{1}
    What is the choice of the geometric decomposition (allocation of
    nodal variables to cell and vertex entities) that leads to the
    maximum possible global continuity of finite element spaces
    defined on the interval $[0,L]$ constructed from the following
    one-dimensional elements $(K,P,N)$.  Justify your answer.
    \BeginParts
    \Part{a}
     $K=[a,b]$, $P$ is linear polynomials, $N=(N_1,N_2)$
      where $N_1[u]=u((a+b)/2)$, $N_2[u]=u'((a+b)/2)$.
      \soln{\similar Since the local space is 2-dimensional, a $C^0$ geometric
        decomposition would require one nodal variable allocated to
        each of the two vertices.  We have
        $u(a)=N_1[u]-(a-b)N_2[u]/2$, and
        $u(b)=N_1[u]+(a-b)N_2[u]/2$. This means that both nodal
        variables are required to determine $u$ at each end of the
        interval. This means it is not possible to allocate one nodal
        variable to each vertex such that the function value can only
        be determined from nodal variables associated with that vertex.
      }
      \exammarks{6}
          \Part{b}
   $K=[a,b]$, $P$ is quadratic polynomials, $N=(N_1,N_2,N_3)$
  where $N_1[u]=u(a)$, $N_2[u]=u(b)$, $N_3[u]=\int_a^b u \diff x$.
  \soln{\similar 
    A $C^1$ geometric decomposition
    would require at least two nodal variables allocated to each vertex,
    so that both the function and the derivative can be determined, but the
    local space is 3 dimensional which does not give enough nodal variables.
    Hence the global space is at most $C^0$. A $C^0$ decomposition allocates
    $N_1$ to the vertex $a$, $N_2$ to the vertex $b$, and $N_3$ to the cell.
    Clearly $u(a)$ can be determined from $N_1$ and $u(b)$ from $N_2$ as
    required.  }
  \exammarks{6}
      \Part{c}
  $K=[a,b]$, $P$ is quadratic polynomials, $N=(N_1,N_2,N_3)$
  where $N_1[u]=u'(a)$, $N_2[u]=u'(b)$,$N_3[u]=u((a+b)/2)$.
  \soln{\similar  The space must be at most $C^0$ by the arguments in the previous part.
    We have $u(a)=N_3[u] + (a-b)(N_1[u]+N_2[u])/4$,
    $u(b)=N_3[u] - (a-b)(N_1[u]+N_2[u])/4$, which means that all three
    variables are required to determine the function values at each vertex.
    By a similar argument to the first part, this means that a $C^0$
    geometric decomposition is impossible.
  }
    \exammarks{7}
\EndParts

\end{Question}
\EndPage

\begin{Question}{2}
\BeginParts
\Part{a}
Consider the finite element $(K,\mathcal{P},\mathcal{N})$, with 
\begin{itemize}
\item $K$ is a non-degenerate triangle,
\item $\mathcal{P}$ is the space of polynomials on $K$ of degree $\leq 1$.
\item $\mathcal{N}=(N_{1},N_{2},N_{3})$,
where 
\[
N_i(u)= \int_{f_i} u\diff x,
\]
where $(f_1,f_2,f_3)$ are the edges of $K$, with $f_1$ joining
vertices $1$ and $2$, $f_2$ joining vertices $2$ and $3$, and $f_3$
joining vertices 3 and 1.
\end{itemize}
Show that $\mathcal{N}$ determines $\mathcal{P}$. \exammarks{10}
\soln{\similar 
  It suffices to show that if $u \in P$, then
   $N_{i}(u)=0$ for all $i$ $\implies u=0$.

   So, we assume that $u\in P$ with $N_{i}(u)=0$, looking to show that
   $u = 0$. $N_i(u)=0$ means that the average of $u$ over the edge
   $f_i$ is zero. Since $u$ is linear on $f_i$, this means that $u$
   vanishes at the midpoint of each edge. These edges can be joined by
   three lines $L_1$, $L_2$, $L_3$, and we iteratively conclude that
   $u$ vanishes on $L_1$ and $L_2$, so that $u=cL_1(x)L_2(x)$, and u
   vanishing on the third vertex not intersected by $L_1$ means that
   $c=0$ (following the usual argument for linear Lagrange elements on
   triangles), and hence $u=0$ everywhere.
 }
 \Part{b} 
  Now consider the finite element $(K,\mathcal{P},\mathcal{N})$, with 
\begin{itemize}
\item $K$ is a non-degenerate triangle,
\item $\mathcal{P}$ is the space of polynomials on $K$ of degree $\leq 2$.
\item $\mathcal{N}=(N_{1,1},N_{1,2},N_{2,1},N_{2,2},N_{3,1},N_{3,2})$,
where 
\[
N_{i,j}(u)= \int_{f_i}\phi_{i,j} u\diff x,
\]
where the edge test functions $\phi_{i,j}$ define
a basis for linear functions restricted to $f_i$ such that $\phi_{i,1}=1$
on vertex $1$ and 0 on vertex $2$, $\phi_{i,2}=1$
on vertex $2$ and 0 on vertex $1$, \emph{etc.}
\end{itemize}
Show that $\mathcal{N}$ does not determine $\mathcal{P}$.
\exammarks{10}
\soln{\unseen We show by counter example. We take the quadratic $q$ function that
  is equal to $1/6$ on each vertex, and $-1/12$ at each edge midpoint
  (this defines a unique quadratic function since these are the nodal
  variables for the standard Lagrange quadratic element, which is
  unisolvent). Consider one of the edges $f_i$, and choose a coordinate $s$
  which is equal to 0 on one end of the edge, and 1 on the other end. On
  that edge, $q|_{f_i}(s)=s^2-s + 1/6$. This function has mean zero, and
  is symmetric, which means that
  \[
  \int_{f_i}\phi(s)q|_{f_i}(s)\diff s=0
  \]
  for any linear function $\phi(s)$, and hence $N_{i,j}(u)=0$, $j=1,2$.
  This means that all the nodal variables vanish when applied to $u$, but
  $u$ is not zero. Hence, $\mathcal{N}$ does not determine $\mathcal{P}$.
} \EndParts
\end{Question}
\EndPage

\begin{Question}{3}
\BeginParts
\Part{a}
Let $b$ be a continuous, coercive bilinear form on $V$, and
$F$ be a continuous linear form on $V$.
Let $u\in V$ solve the linear variational problem
\[
b(u,v) = F(v) \quad \forall v \in V.
\]
Let $V_h$ be a finite dimensional subspace of $V$,
and let $u_h \in V$ solve the Galerkin approximation
\[
b(u_h,v) = F(v) \quad \forall v \in V_h.
\]
Show that
\[
b(u-u_h,v) = 0, \quad \forall v \in V_h.
\]
\exammarks{4}
\soln{\bookwork
  Since $V_h \subset V$, we can take $v\in V_h$ in the variational
  problem for $u$, to get
  \[
  b(u,v) = F(v) \quad \forall v \in V_h.
  \]
  Then, subtracting the Galerkin approximation, we have
  \[
  b(u,v) - b(u_h,v) = 0 \quad \forall v \in V_h.
  \]
  Finally, from bilinearity, we have
    \[
  b(u-u_h,v) = 0 \quad \forall v \in V_h.
  \]
}
\Part{b}
Hence, show that
\[
\|u-u_h\|_V \leq \frac{M}{\gamma}\min_{v\in V_h}\|u-v\|_V,
\]
where $\gamma$ and $M$ are the coercivity and continuity constants for
$b$ respectively.
\exammarks{4}
\soln{\bookwork
\begin{align*}
  \gamma \|u-u_h\|_V^2 & \leq b(u-u_h,u-u_h), \quad \mbox{[coercivity]} \\
  & = b(u-u_h,u-v+v-u_h) \quad \mbox{for any }v\in V_h, \\
  & = b(u-u_h,u-v) + \underbrace{b(u-u_h,v-u_h)}_{=0\mbox{[by previous part]}},
  \quad \mbox{[bilinearity]} \\
  & \leq M\|u-u_h\|_V\|u-v\|_V. \quad \mbox{[continuity]}
\end{align*}
Dividing both sides by $\gamma\|u-u_h\|_V$ gives
\[
\|u-u_h\|_V \leq \frac{M}{\gamma}\|u-v\|_V.
\]
Since the left-hand side is independent of the choice of $v$, we
can minimise over $v\in V_h$ to get the result.
}
\Part{c}Consider the variational problem of finding $u\in H^1([0,1])$
such that
\[
\int_0^1 vu + v'u' \diff x = \int_0^1 vx \diff x + v(1) - v(0),
\quad \forall v \in H^1([0,1]).
\]
After dividing the interval $[0,1]$ into $N$ equispaced cells and forming a $P1$ $C^0$ finite
element space $V_N$, the error $\|u-u_h\|_{H^1}=0$ for any $N>0$.

Explain why this is expected.
\exammarks{6}
\soln{\similar
  The problem has the solution $u(x)=x$. Hence, $u\in V_N$ for any $N$,
  and $\min_v \|v-u\|_{H^1([0,1])}=0$, hence $\|u-u_h\|_{H^1([0,1])}=0$
  from the previous result.
}
\Part{d}
Let $ \mathring{H}^1([0,1])$ be the subspace of $H^1([0,1])$ such that $u(0)=0$.
Consider the variational problem of finding $u \in \mathring{H}^1([0,1])$  with
\[
\int_0^1 v'u' \diff x = \int_0^{1/2} v \diff x, \quad \forall v \in \mathring{H}([0,1]).
\]
The interval $[0,1]$ is divided into $3N$ equispaced cells (where $N$
is a positive integer). After forming a $P1$ $C^0$ finite element
space $V_N$, the error $\|u-u_h\|_{H^1}$ is found not to converge to
zero. Explain why this is expected?
\exammarks{6}
\soln{\unseen
  The solution is
  \[
  u(x) = \left\{
  \begin{array}{c c}
    \frac{x-x^2}{2} & x<1/2, \\
    \frac{1}{8} & \mbox{otherwise.}
  \end{array}
  \right.
  \]
  This solution is not in $H^2$, and so the usual interpolation estimate
  is not expected.
  }
\EndParts
\end{Question}
\EndPage

\begin{Question}{4}
The inhomogeneous Helmholtz equation in two dimensions is given by
\begin{equation}
\alpha(x)u - \nabla^2 u = f, \quad
\pp{u}{n} = 0 \mbox{ on }\partial\Omega,
\label{eq:helm}
\end{equation}
where $\partial\Omega$ is the boundary of the problem domain $\Omega$,
and $\alpha(x)$ is a $C^\infty(\Omega)$ function with bounds $1 \leq
\alpha(x) \leq 2$ for all $z\in \Omega$.
\BeginParts
\Part{a}
Write down a variational formulation for this problem, in the form
\[
a(u, v) = F (v),
\quad \forall v \in H^1(\Omega),
\]
and show that if $u$ solves the variational formulation, and $u\in
H^2(\Omega)$ then $u$ solves \eqref{eq:helm} in an appropriate sense.
\exammarks{6}
\soln{\bookwork
  We take
  \[
  a(u,v) = \int_\Omega \alpha uv + \nabla u\cdot \nabla v \diff x,
  \quad F(v) = \int_\Omega vf\diff x.
  \]
  Taking $v\in C_0^\infty(\Omega)$, we have enough regularity for
  integration by parts, and
  \[
  \int_\Omega v(\alpha u - \nabla^2 u - f)\diff x = 0.
  \]
  Picking a sequence of $v$s that converge to $\alpha u - \nabla^2 u - f$,
  and passing to the limit gives $\|\alpha u - \nabla^2u -f \|_{L^2(\Omega)}$,
  i.e. $\alpha u - \nabla^2u = f$ in $L^2(\Omega)$. Returning to
  the variational form and using this fact gives
  \begin{align*}
    0 & = \int_\Omega \alpha uv + \nabla u\cdot \nabla v - vf\diff x,
    \\ & = \int_\Omega \alpha uv + \nabla u\cdot \nabla v - v(\alpha u
    - \nabla^2 u)\diff x, \\
    & = \int_{\partial\Omega} \pp{u}{n} v \diff S, \quad \forall v \in H^1(\Omega),
  \end{align*}
  after using integration by parts. We may choose $v = \pp{u}{n}$ from
  the trace theorem, and hence $\|\pp{u}{n}\|_{L^2(\Omega)}=0$ i.e.
  $\pp{u}{n}=0$ in $L^2(\Omega)$.
  }
\Part{b} Show that $a(\cdot,\cdot)$ is continuous and coercive.
\exammarks{6}
\soln{\similar
\[
|a(u,v)| = |\int_\Omega \alpha uv + \nabla u\cdot \nabla v \diff x|
\leq 2|\int_\Omega uv + \nabla u\cdot \nabla v \diff x| = 2|(u,v)|_{H^1(\Omega)}
 \leq \|u\|_{H^1(\Omega)}\|v\|_{H^1(\Omega)},
 \]
 by the Schwartz inequality. Hence $a$ is continuous with continuity constant
 $2$.
 \[
 a(u,u) = \int_\Omega \alpha u^2 + |\nabla u|^2 \diff x
 \geq\int_\Omega u^2 + |\nabla u|^2 \diff x
= \|u\|_{H^1(\Omega)}^2,
\]
hence $a$ is coercive with coercivity constant $1$.}
\Part{c} Hence, show that the linear Lagrange finite element approximation
satisfies
\[
\|u-u_h\|_{H^1(\Omega)} \leq Ch\|u\|_{H^2(\Omega)}.
\]
for $C > 0$, independent of $u$. (You may make use of the
approximation theory estimate
\[
\|u-I_hu\|_{H^1(\Omega)} \leq \hat{C}h\|u\|_{H^2(\Omega)}.
\]
for $\hat{C} > 0$, independent of $u$, where $I_h$ is the nodal interpolation
operator $I_h:H^2(\Omega)\to V_h$, where $V_h$ is the finite element space
with mesh parameter $h$, and any other results from lectures.)
\exammarks{8}
\soln{\unseen
  From C\`ea's Lemma (also proven in Q3) we have
  \[
  \|u-u_h\|_{H^1(\Omega)} \leq \frac{M=2}{\gamma=1}\min_{v\in V_h}
  \|v-u\|_{H^1(\Omega)} \leq 2\|I_hu - u \|_{H^1(\Omega)}
  \leq 2\hat{C}h\|u\|_{H^2(\Omega)}.
  \]
  }
\EndParts
\end{Question}
\EndPage
\label{ptotal}
\end{document}
